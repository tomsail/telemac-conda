\renewcommand{\labelitemi}{$\triangleright$}

\chapter{Frazil formation in a reservoir}
%
% - Purpose & Problem description:
%     These first two parts give reader short details about the test case,
%     the physical phenomena involved and specify how the numerical solution will be validated
%
\section{Purpose}

The purpose of this test case is to validate the implementation of the multiclass frazil ice model.
In order to validate implementation each process i.e. thermal growth, 
seeding, secondary nucleation, flocculation and the impact of salinity are tested independently.
This test case demonstrates that \khione is capable of modeling 
supercooling and frazil ice growth with several classes of frazil particles.

\section{Description}

\subsection{Mesh, bathymetry and boundary conditions}

The bathymetry is set to zero at all points of the domain. The mesh and the boundary conditions are represented in Figure \ref{fig:res_mesh}.

\begin{figure}[H]
    \begin{center}
        \includegraphicsmaybe{[width=0.5\textwidth]}{../img/mesh.png}
    \end{center}
    \caption{Domain mesh}
    \label{fig:res_mesh}
\end{figure}

\subsection{Initial conditions}

The initial condition is a constant elevation of 1 m and no velocity. The value of tracer in \telemac{2D} are set to 0 at the initial time. For thermal growth, the initial volume fraction of frazil is nil, and for melting, it is set to 0.001.

\subsection{Atmospheric drivers}

For this test case, linear heat exchanges are used i.e. within the \khione steering file:
\begin{itemize}
	\item\textit{ATMOSPHERE-WATER EXCHANGE MODEL = 0}
\end{itemize}
A contant air temperature of $-25^\circ C$ is applied in the case of growth and $+20^\circ C$ in the case of melting.

\subsection{Physical parameters}

Physical properties related to frazil growth rate are defined in the \khione steering file. 
For both the mono-class and multi-class models, the turbulent parameters are set to constants
and defined with the keywords:
\begin{itemize}
	\item\textit{MODEL FOR ESTIMATION OF TURBULENCE PARAMETERS = 0}
	\item\textit{CONSTANT TURBULENCE PARAMETERS = 9.6D-4; 1.2D-3; 6.912D-05; 8.76D-2}
\end{itemize}
i.e. turbulent kinetic energy ($k$),
turbulent dissipation rate ($\varepsilon$), turbulent viscosity ($\nu_t$) and 
turbulent instensity ($\alpha_t$) respectively.
The Nusselt number is then computed within \khione depending on these parameters, 
with \textit{MODEL FOR THE NUSSELT NUMBER = 2}.

\subsubsection{Mono-class model parameters}
With the mono-class model, frazil ice related parameters are set to default.
The initial seeding is defined with a minimum threshold for the number of crystals per unit volume.
\begin{itemize}
  \item\textit{MODEL FOR FRAZIL SEEDING = 1}
\end{itemize}
The parameters \textit{MINIMUM NUMBER OF FRAZIL CRYSTALS} is set to default values.

\subsubsection{Multi-class model parameters}
With the mono-class model, frazil ice crystal' sizes are defined as:
\begin{itemize}
	\item\textit{NUMBER OF CLASSES FOR SUSPENDED FRAZIL ICE = 10}
	\item\textit{FRAZIL CRYSTALS RADIUS = 1.E-4 ; 2.E-4 ; ... ; 9.E-4 ; 10.E-4}
\end{itemize}
The initial seeding is either defined 
with a minimum threshold for the number of crystals or a seeding rate.
In the second case, the seeding rate is only applied on the first class.
The parameters \textit{MINIMUM NUMBER OF FRAZIL CRYSTALS} and \textit{FRAZIL SEEDING RATE}
\begin{itemize}
	\item\textit{MINIMUM NUMBER OF FRAZIL CRYSTALS = 3200.}
	\item\textit{ FRAZIL SEEDING RATE = 8.E3 ; 0. ; 0. ; 0. ; 0.; 0. ; 0. ; 0. ; 0. ; 0.}
\end{itemize}
Note that in case of multiple classes the parameters \textit{MINIMUM NUMBER OF FRAZIL CRYSTALS}
stands for the total number of crystals, and is equally shared between all classes. 
In case of $10$ classes \textit{MINIMUM NUMBER OF FRAZIL CRYSTALS = 3200.} means
$320$ min. crystals for each class.
The secondary nucleation and the flocculation processes are defined as follows:
\begin{itemize}
	\item\textit{MODEL FOR THE BUOYANCY VELOCITY = 4}
	\item\textit{MODEL FOR THE SECONDARY NUCLEATION = 1}
	\item\textit{SECONDARY NUCLEATION NMAX PARAMETER = 1.E6}
	\item\textit{MODEL FOR THE FLOCCULATION AND BREAKUP = 1}
	\item\textit{FLOCCULATION AFLOC PARAMETER = 1.E-4}
\end{itemize}
The parameters $n_{max}$ and $a_{floc}$ parameters are taken as calibration parameters 
that are tuned to fit with the experimental data.

These default parameters defined above are modified in the tests to evaluate the response of the model
for each process.

\subsection{Numerical parameters}
The time step is set to $0.1$ s and the duration is set to $800$ s for the thermal growth cases and $1600$ s for melting tests.

\section{Results}

The result of the monoclass and multiclass test are presented in Figures \ref{fig:monoclass_TCf}.
\begin{figure}[H]
    \begin{center}
        \includegraphicsmaybe{[width=0.49\textwidth]}{../img/monoclass_TCf.png}
        \includegraphicsmaybe{[width=0.49\textwidth]}{../img/multiclass_TCf.png}
    \end{center}
    \caption{Temperature and frazil volume fraction for monoclass (left) and multiclass (right) models}
    \label{fig:monoclass_TCf}
\end{figure}
Figure \ref{fig:multiclass_TCf_melt} shows the results for the multiclass melting case.
\begin{figure}[H]
    \begin{center}
        \includegraphicsmaybe{[width=0.5\textwidth]}{../img/multiclass_TCf_melt.png}
    \end{center}
    \caption{Temperature and frazil volume fraction for multiclass run in case of melting}
    \label{fig:multiclass_TCf_melt}
\end{figure}

The figure \ref{fig:seed} show the influence of the chosen seeding model in the frazil volume fraction and temperature values.
\begin{figure}[H]
    \begin{center}
        \includegraphicsmaybe{[width=0.45\textwidth]}{../img/seed_temp.png}
        \includegraphicsmaybe{[width=0.45\textwidth]}{../img/seed_fra.png}
    \end{center}
    \caption{Temperature and frazil volume fraction for multiclass run depending on the seeding model chosen}
    \label{fig:seed}
\end{figure}

The figure \ref{fig:seeding_rate} show the influence of the chosen seeding rate in the frazil volume fraction and temperature values.
\begin{figure}[H]
    \begin{center}
        \includegraphicsmaybe{[width=0.45\textwidth]}{../img/seeding_rate_temp.png}
        \includegraphicsmaybe{[width=0.45\textwidth]}{../img/seeding_rate_fra.png}
    \end{center}
    \caption{Temperature and frazil volume fraction for multiclass run depending on the seeding rate}
    \label{fig:seeding_rate}
\end{figure}

The figure \ref{fig:min_part_fra} show the influence of the minimum particle value given for the seeding on the frazil volume fraction and temperature in the multiclass model.
\begin{figure}[H]
    \begin{center}
        \includegraphicsmaybe{[width=0.45\textwidth]}{../img/min_part_temp.png}
        \includegraphicsmaybe{[width=0.45\textwidth]}{../img/min_part_fra.png}
    \end{center}
    \caption{Temperature and frazil volume fraction for multiclass run depending on the minimal particle given for seeding}
    \label{fig:min_part_fra}
\end{figure}

The figure \ref{fig:floc_model_fra} show the influence of the activation of the floculation model in the temperature and frazil volume fraction values for the multiclass model.
\begin{figure}[H]
    \begin{center}
        \includegraphicsmaybe{[width=0.45\textwidth]}{../img/floc_model_temp.png}
        \includegraphicsmaybe{[width=0.45\textwidth]}{../img/floc_model_fra.png}
    \end{center}
    \caption{Temperature and frazil volume fraction for multiclass run whether the floculation model is activated or not}
    \label{fig:floc_model_fra}
\end{figure}

The figure \ref{fig:floc_alpha_fra} show the sensitivity of the model to the variation of the afloc parameter in the floculation model.
\begin{figure}[H]
    \begin{center}
        \includegraphicsmaybe{[width=0.45\textwidth]}{../img/floc_alpha_temp.png}
        \includegraphicsmaybe{[width=0.45\textwidth]}{../img/floc_alpha_fra.png}
    \end{center}
    \caption{Temperature and frazil volume fraction for multiclass run depending on the chosen afloc parameter}
    \label{fig:floc_alpha_fra}
\end{figure}

The figure \ref{fig:nmax_fra} show the sensitivity of the model to the variation of the nmax parameter in the nucleation model.
\begin{figure}[H]
    \begin{center}
        \includegraphicsmaybe{[width=0.45\textwidth]}{../img/nmax_temp.png}
        \includegraphicsmaybe{[width=0.45\textwidth]}{../img/nmax_fra.png}
    \end{center}
    \caption{Temperature and frazil volume fraction for multiclass run depending on the chosen nmax parameter}
    \label{fig:nmax_fra}
\end{figure}

The figure \ref{fig:nuc_model_fra} show the influence of the activation of the nucleation model in the temperature and frazil volume fraction values for the multiclass model.
\begin{figure}[H]
    \begin{center}
        \includegraphicsmaybe{[width=0.45\textwidth]}{../img/nuc_model_temp.png}
        \includegraphicsmaybe{[width=0.45\textwidth]}{../img/nuc_model_fra.png}
    \end{center}
    \caption{Temperature and frazil volume fraction for multiclass run whether the secondary nucleation model is activated or not}
    \label{fig:nuc_model_fra}
\end{figure}

The figure \ref{fig:buo_fra} show the influence of the buoyancy velocity model chosen in the volume fraction and temperature values.
\begin{figure}[H]
    \begin{center}
        \includegraphicsmaybe{[width=0.45\textwidth]}{../img/buo_temp.png}
        \includegraphicsmaybe{[width=0.45\textwidth]}{../img/buo_fra.png}
    \end{center}
    \caption{Temperature and frazil volume fraction for multiclass depending on the buoyancy model chosen}
    \label{fig:buo_fra}
\end{figure}

\section{Conclusions}

The model is able to reproduce the supercooling and formation of frazil with several classes. 
A small analysis shows that the equilibrium between the secondary nucleation parameter and the
floculation model parameter must be found in order to have a good repartition of the frazil classes in the total volume fraction.

\renewcommand{\labelitemi}{\textbullet}
