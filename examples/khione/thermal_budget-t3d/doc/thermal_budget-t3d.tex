\renewcommand{\labelitemi}{$\triangleright$}

\chapter{Thermal budget - 3D}
%
% - Purpose & Problem description:
%     These first two parts give reader short details about the test case,
%     the physical phenomena involved and specify how the numerical solution will be validated
%
\section{Purpose}
This test case demonstrates that \khione is able of correctly model heat exchanges at the free surface depending on atmsopheric data. In this test case, atmospheric input data is taken from Massena, NY, USA, near the St Laurence River.

\section{Model description}

\subsection{Geometry and mesh}
The domain is a square box, centred on $(0;0)$, with a size of 4 m x 4 m. The box has a flat bottom, set at
0 m. The mesh of the domain was created with a uniform density of about 0.35 m, resulting in a mesh with 332
elements and 193 vertices. (see figure \ref{fig:thermal_mesh}). The considered mesh is extruded with 2 planes
in the vertical dimension.

\begin{figure}[H]
    \begin{center}
        \includegraphicsmaybe{[scale=0.45]}{../img/mesh.png}
    \end{center}
    \caption{Domain mesh}
    \label{fig:thermal_mesh}
\end{figure}


\subsection{Initial conditions}
Water is at rest and remains so during the simulation. Water depth is set to 1~m.
Water temperature is initially set at $1^\circ$C.

\subsection{Boundary conditions}
Boundaries are considered as solid walls with perfect slip conditions.

\subsection{Atmospheric drivers}

There are no other drivers than the atmospheric exchanges. Temporal variations in air temperature, cloud cover, dew temperature, visibility, snow, rain and wind speed are provided through the ASCII file within the \telemac{3D} steering file. Also, in order to refer to the angle of the sun, the latitude of the model is provided in the \telemac{3D} seering file:
\begin{itemize}
    \item\textit{LATITUDE OF ORIGIN POINT = 44.73}
    \item\textit{ASCII ATMOSPHERIC DATA FILE =} 't3d\_meteo\_stlaurent.lqd'
\end{itemize}
For constant atmospheric data over time, values can also be set within the \khione steering file:
\begin{itemize}
    \item\textit{AIR TEMPERATURE      = -10.0}
    \item\textit{DEWPOINT TEMPERATURE =   0.0}
    \item\textit{CLOUD COVER          =   0.0}
    \item\textit{VISIBILITY           =   1.E9}
    \item\textit{RAIN                 =   0.0}
    \item\textit{WIND XY-COMPONENTS   =   5.;3.}
    \item\textit{RELATIVE MODEL ELEVATION FROM MEAN SEA LEVEL = 80.}
\end{itemize}

For the full thermal budget the heat exchange model is set to 4 within \khione streering file:
\begin{itemize}
    \item\textit{ATMOSPHERE-WATER EXCHANGE MODEL = 2}
\end{itemize}
With this model, the net heat flux received by water trought the water/air interface is expressed as:
\begin{equation}
\phi = \phi_R + C_B\phi_B + C_E\phi_E + C_H\phi_H + C_P\phi_P
\end{equation}
where $\phi_R$, $\phi_B$, $\phi_E$, $\phi_H$ and $\phi_P$ are the net solar radiation flux, the effective back
radiation flux, the evaporation heat transfert, the conductive heat transfert and the precipitation heat
transfert respectively. $C_B$, $C_E$, $C_H$ and $C_P$ are calibration coefficient, with default values of $1.$,
that are defined in the \khione steering file via the folling keywords:
\begin{itemize}
    \item\textit{COEFFICIENT FOR CALIBRATION OF BACK RADIATION = 1.}
    \item\textit{COEFFICIENT FOR CALIBRATION OF EVAPORATIVE HEAT TRANSFERT = 1.}
    \item\textit{COEFFICIENT FOR CALIBRATION OF CONDUCTIVE HEAT TRANSFERT = 1.}
    \item\textit{COEFFICIENT FOR CALIBRATION OF PRECIPITATION HEAT TRANSFERT = 1.}
\end{itemize}

\subsection{Physical parameters}
Within the \khione steering file, water and air specific heat are set to 1002 J.kg$^{-1}$.K$^{-1}$ and 4180 J.kg$^{-1}$.K$^{-1}$.

\subsection{Numerical parameters}
The time step is set to 300 s and the number of timestep is set to $288$ which leads to a simulation time of 24 h.

~\newline
\section{Results}
On Figure \ref{fig:thermal_fluxes} and \ref{fig:thermal_temp} thermal fluxes and water temperature evolution are plotted against time with the comprehensive thermal budget model.
For comparison, the thermal fluxes at the free surface obtained with the simplified linear model are plotted on Figure \ref{fig:thermal_fluxes_linear}.

\begin{figure}[H]
    \begin{center}
        \includegraphicsmaybe{[width=0.8\textwidth]}{../img/thermal_fluxes_0.png}
    \end{center}
    \caption{Thermal fluxes at the free surface with the full budget model}
    \label{fig:thermal_fluxes}
\end{figure}

\begin{figure}[H]
    \begin{center}
        \includegraphicsmaybe{[width=0.8\textwidth]}{../img/thermal_fluxes_1.png}
    \end{center}
    \caption{Thermal fluxes at the free surface with the default linear budget model}
    \label{fig:thermal_fluxes_linear}
\end{figure}

\begin{figure}[H]
    \begin{center}
        \includegraphicsmaybe{[width=0.8\textwidth]}{../img/temperature.png}
    \end{center}
    \caption{Evolution of the water temperature}
    \label{fig:thermal_temp}
\end{figure}

The stratification of the temperature in the water column is shown on Figure \ref{fig:temp1} and \ref{fig:temp2}. One can see that the flux is applied at the free surface boundary so the temperature is higher in the top of the water column at the middle of the simulation when the air is warming up the water. When the air is cooling down the water at the end of the simulation, its temperature is lower at the top of the water column.

\begin{figure}[H]
    \begin{center}
        \includegraphicsmaybe{[width=0.8\textwidth]}{../img/temperature_T1.png}
    \end{center}
    \caption{Vertical temperature distribution at the middle of the simulation}
    \label{fig:temp1}
\end{figure}

\begin{figure}[H]
    \begin{center}
        \includegraphicsmaybe{[width=0.8\textwidth]}{../img/temperature_T2.png}
    \end{center}
    \caption{Vertical temperature distribution at the end of the simulation}
    \label{fig:temp2}
\end{figure}

\section{Conclusions}

The water remains at rest and the temperature evolves as a result of changes in the atmospheric conditions (air temperature, solar radiation, etc.). The model is also able to represent the vertical distribution of the temperature in the water column.


\renewcommand{\labelitemi}{\textbullet}
