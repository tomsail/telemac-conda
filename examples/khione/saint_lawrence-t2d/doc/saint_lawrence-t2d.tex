\renewcommand{\labelitemi}{$\triangleright$}

\chapter{Saint Lawrence}

\section{Purpose}

The border ice formation in the reach of St. Lawrence River from Ogdensburg to Power Dam is
simulated to test the applicability of the border ice module. This reach is an ideal channel to
examine the border ice module because of the complex geometry with number of islands and
large variation of flow depth. 
This test case also demonstrates that \khione is capable of modeling frazil ice formation,
transport accumulation at the free surface in a real life configuration.

\section{Description}

\subsection{Geometry, mesh and bathymetry}

The domain upstream boundary is set to Ogdensburg and the downstream is set to the Power Dam.
%The mesh of the domain was created with a uniform density and with $23960$ elements and $12600$ vertices (cf. figure \ref{fig:mesh}).

\begin{figure}[H]
    \begin{center}
        \includegraphicsmaybe{[width=0.9\textwidth]}{../img/mesh.png}
    \end{center}
    \caption{Domain mesh}
    \label{fig:mesh}
\end{figure}

\begin{figure}[H]
    \begin{center}
        \includegraphicsmaybe{[width=\textwidth]}{../img/bottom.png}
    \end{center}
    \caption{Bathymetry}
    \label{fig:bathy}
\end{figure}


\subsection{Initial conditions}

A hydrodynamics steady state condition is established first separately to reach a constant and uniform elevation and a constant and uniform discharge.
Water temperature is initially set at $0.0^\circ C$.

\subsection{Atmospheric drivers and boundary conditions}

In Figure \ref{fig:mto_data} the observed discharge and air temperature are given for the winter of 2019.
Several freeze-up periods are highlighted.

\begin{figure}[H]
    \begin{center}
        \includegraphicsmaybe{[width=0.45\textwidth]}{img/temperature_02-2019.png}
        \includegraphicsmaybe{[width=0.45\textwidth]}{img/flowrate_02-2019.png}
    \end{center}
    \caption{Measured air temperature at Station OBGN6 (8311030 Ogdensburg) and measured discharge during winter 2019 at Robert Moses-Robert H. Saunders power dam on Lake St. Lawrence (source: USGS)}
    \label{fig:mto_data}
\end{figure}

There are 2 open boundaries as described in Figure \ref{fig:mesh}.
The solid boundaries are considered as solid walls with perfect slip conditions.
A constant water surface elevation of $74.40$ m is considered at the upstream  boundary
and a constant discharge of $6400$m$^3$/s is imposed at the downstream boundary.

For this test case, the full heat exchanges are used i.e. within the \khione steering file:
\begin{itemize}
	\item\textit{ATMOSPHERE-WATER EXCHANGE MODEL = 1}
\end{itemize}
Air temperature is set to a constant of $-20^{\circ}$ during the simulation.

\subsection{Physical parameters}

In order to balance the slope and the discharge through the flume, a Manning friction law is used, with a coefficient of $0.025$.

To activate the simulation of border ice processes, the following keywords are added to the \khione steering file:
\begin{itemize}
	\item\textit{BORDER ICE COVER = YES}
  \item\textit{ICE COVER IMPACT ON HYDRODYNAMIC = YES}
\end{itemize}


\subsection{Numerical parameters}
The time step is set to $1s$ and the number of timestep is set to $86400$ which leads to a simulation time of one day.

\section{Results}

Figure \ref{fig:velocity} shows the simulated steady state water velocity. Figure \ref{fig:observed_border_ice_cover} and Figure \ref{fig:simulated_border_ice_cover} show the
good agreement between observed and simulated border ice distribution in the St. Lawrence
River.

\begin{figure}[H]
    \begin{center}
        \includegraphicsmaybe{[width=\textwidth]}{../img/water_depth_1.png}
    \end{center}
    \caption{Simulated steady state water depth.}
    \label{fig:water_depth}
\end{figure}

\begin{figure}[H]
    \begin{center}
        \includegraphicsmaybe{[width=\textwidth]}{../img/velocity_1.png}
    \end{center}
    \caption{Simulated steady state water velocity.}
    \label{fig:velocity}
\end{figure}

\begin{figure}[H]
    \begin{center}
        \includegraphicsmaybe{[width=0.8\textwidth]}{img/observed_border_ice_cover.png}
    \end{center}
    \caption{Observed typical border ice distribution during freeze up in the St. Lawrence River.}
    \label{fig:observed_border_ice_cover}
\end{figure}

%\begin{figure}[H]
%    \begin{center}
%        \includegraphicsmaybe{[width=\textwidth]}{../img/ice_type_1.png}
%    \end{center}
%    \caption{Simulated border ice distribution in the St. Lawrence River.}
%    \label{fig:simulated_border_ice_cover}
%\end{figure}

\begin{figure}[H]
    \begin{center}
        \includegraphicsmaybe{[width=\textwidth]}{../img/ice_titot_1.png}
    \end{center}
    \caption{Simulated border ice thickness in the St. Lawrence River.}
    \label{fig:simulated_border_ice_cover}
\end{figure}


\section{Conclusions}
This test case shows the ability of \khione to model the formation of border ice along a river during a freeze-up period.

\renewcommand{\labelitemi}{\textbullet}
