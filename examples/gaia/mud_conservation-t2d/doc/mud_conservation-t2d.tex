\chapter{Conservation (mud\_conservation-t2d)}
%

% - Purpose & Problem description:
%     These first two parts give reader short details about the test case,
%     the physical phenomena involved and specify how the numerical solution will be validated
%
\section{Purpose}
The aim of this test is to check the mass conservation of suspended sediments for unsteady flow conditions, as well as bed mass conservation. The phenomena consists in a spreading circular wave in which a suspended sediment is released.

\section{Description}

\subsection{Geometry and Mesh}
%
% copied by gouttedo (t2d) since same geo
\begin{figure}[h]
\begin{center}
  \includegraphicsmaybe{[width=0.5\textwidth]}{../img/Mesh.png}
\end{center}
\caption{Mesh}
\label{fig:cons_mesh}
\end{figure}

The domain is square with a size of  20.1~m x 20.1~m with a flat bottom.
The domain is meshed with 8978 triangular elements and 4624 nodes. Triangles
are obtained by dividing rectagular elements on their diagonals. The mean size
of obntained triangles is about 0.3~m (see figure \ref{fig:cons_mesh}).
%
\subsection{Initial conditions}
%
% copied by gouttedo (t2d) since same hydro
The fluid is initially at rest with a Gaussian free surface in the centre of a
square domain (see Figure \ref{fig:water_init}, cf. gouettedo test case of \telemac{2D} ). Water depth is given by
$ H= 2.4 \left(1.0+exp \left( \frac{-\left[ (x-10)^2+( y-10)^2\right]}{ 4}\right)\right) $

\begin{figure}[H]
\begin{center}
  \includegraphicsmaybe{[width=0.9\textwidth]}{../img/InitialElevation.png}
\end{center}
\caption{Mud conservation: initial elevation}
\label{fig:water_init}
\end{figure}

The suspended sediment is equal to 0 everywhere except in a central zone of the domain (where the Gaussian free surface is defined)
 where it is equal to 1:
$C=1 ~\text{if}~ (x-10.05)^2+(y-10.05)^2<4^2$

\begin{figure}[H]
\begin{center}
  \includegraphicsmaybe{[width=0.9\textwidth]}{../img/InitialConcentration.png}
\end{center}
\caption{Mud conservation: initial sediment concentration}
\label{fig:mudcons_init}
\end{figure}
%
\subsection{Boundary conditions}
%
Boundaries are solid walls with perfect slip conditions.
%

% - Numerical parameters:
%     This part is used to specify the numerical parameters used
%     (adaptive time step, mass-lumping when necessary...)
%
\subsection{Physical parameters}
%
% copied by gouttedo (t2d) since same hydro
The physical parameters for hydrodynamics used for this case are the following:
\begin{itemize}
\item Friction: Strickler formula with k$_s$ = 40 m$^{1/3}$/s
\item Turbulence: Constant viscosity equal to zero (no diffusion of velocities)
\end{itemize}
%
The sediment is cohesive with a constant diameter equal to 0.06 mm and density equal to 1600 kg/m$^3$. The settling velocity is set to 0.01 m/s and the Partheniades constant is set to 0.001 kg/m$^2$/s .
%
\subsection{Numerical parameters}

% - Results:
%     We comment in this part the numerical results against the reference ones,
%     giving understanding keys and making assumptions when necessary.
%
%
\section{Results}
The following figures show time evolutions of the bed lost mass and the relative error of the bed layer: errors are very small.

% Figure - bed lost mass timeseries
\begin{figure} [h]
\centering
\includegraphicsmaybe{[scale=0.5]}{../img/lost_mass_01.png}
\caption{Lost mass evolution}\label{bedmasserr}
\end{figure}
\begin{figure} [h]
\centering
\includegraphicsmaybe{[scale=0.5]}{../img/re_ial_mass_01.png}
\caption{Relative mass error of bed layer}\label{relmasserr_ial}
\end{figure}
% A figure which shows that suspended mass is conserved is necessary but get_tracer_mass_profile is only for 3d cases

\section{Conclusion}
This test shows that mass is conserved for suspended sediment and bed. It is a very simple configuration which allows to quick code debugging and testing (cf. \gaia{} steering file where there are other commented configurations).
%


