\chapter{Porous}
%
% - Purpose & Problem description:
%     These first two parts give reader short details about the test case,
%     the physical phenomena involved and specify how the numerical solution will be validated
%
\section{Purpose}
%
This test case has been created to test the dissipation due to a porous media.
That is only a test that checks non regression solution. In order to test the
subroutine of the trunk the test has not any fortran user. This implies to have
porous media in all the domain of calculation.  


%
\section{Description of the problem}
%
We simulate the dissipation due to porous media in a square. The domain is a
square defined by the points (0,0) and (100,100). Since we are using the domain
set in {\it qporos}, The porous media is a square between points (40,39) and
(50,61).  


\section{Physical parameters}
Simulations were carried out with a water depth h = 2.0~m and a wave heigth of
0.56~m at the incident wave boundary. The media defined in {\it qporos} has a
porosity rate of 0.8 a damping mass coefficient  of 2.5, and a linear friction
coefficient of 1.

\section{Geometry and Mesh}
The computational domain was composed of a flat (slope = 0.0) 2D grid with an
aspect ratio of 1 (cross-shore direction):10 (along shore direction).
The calculation grid size was set as 2.0~m in the wave propagation direction.
There are 906 nodes and 1500 triangles.

The area where Porous media is taken into account is defined by the file
{\it Zone.txt'}

\begin{figure} [!h]
\centering
\includegraphicsmaybe{[width=0.85\textwidth]}{../img/mesh.png}
 \caption{Mesh of the domain.}
\label{meshPorous}
\end{figure}
%
\section{Initial and Boundary Conditions}
%
The initial  significative heigth is 0.015~m with an initial frequency of 1
Hz and an initial peak factor of 3.3 and a mean direction of $90^{\circ}$ and a
initial directionnal spread of 0.
The boundary conditions are given by a jonswap spectrum with a boundary pic
factor of 3.3Hz a significative heigth of 0.5656~m and a boundary frequency pic
of 1~Hz.

% - Numerical parameters:
%     This part is used to specify the numerical parameters used
%     (adaptive time step, mass-lumping when necessary...)
%
%
\section{Numerical parameters}
%
The time step was of 1s and the duration of the computation of 1200~s. The
spectro-angular mesh has 36 directions and 6 frequencies. The frequential
ratio was of 1.01 and the minimum frequency of 0.9705~Hz.

\section{Results}
We can see on Figure \ref{resPorous} the impact of porous media compared to
Figure \ref{ressans}
\begin{figure} [!h]
\centering
\includegraphicsmaybe{[width=0.6\textwidth]}{../img/results.png}
 \caption{Wave heigth hm0 with porous media.}
\label{resPorous}
\end{figure}
\begin{figure} [!h]
\centering
\includegraphicsmaybe{[width=0.6\textwidth]}{../img/sans.png}
 \caption{Wave heigth hm0 without porous media.}
\label{ressans}
\end{figure}
