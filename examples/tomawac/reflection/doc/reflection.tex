\chapter{Reflection}
\section{Purpose}
This test case is an example of reflection effect with \tomawac

\section{Description of the problem}
The domain is homogeneous with the upper and the bottom face reflective.  


\section{Geometry and Mesh}
The domain is a rectangle of 400~m by 100~m.
%The python that created the mesh is in file {\it rectangle.py}

\begin{figure} [!h]
\centering
\includegraphicsmaybe{[width=0.95\textwidth]}{../img/mesh.png}
\caption{Mesh of the domain.}
\label{reflection_mesh}
\end{figure}

\section{Initial and Boundary Conditions}
The waves are coming from the right boundary with an angle of 300~$^{\circ}$, a wave height of 0.2 and a peak period of 1.88~s.

\section{Results}
Seeing the result figure \ref{reflection_result}, the reflection seems to work correctly.
\begin{figure} [!h]
\centering
\includegraphicsmaybe{[width=0.95\textwidth]}{../img/results.png}
\caption{Wave height hm0.}
\label{reflection_result}
\end{figure}

\section{Test of rotated boundary}
In order to check that the reflection condition works for any boundary (not parallel with axes like in previous section), we test the same configuration on a rotated geometry of 30$^{\circ}$ see figure \ref{reflection_mesh2}
%(the python that created the mesh is in file {\it satourne.py})
\begin{figure} [!h]
\centering
\includegraphicsmaybe{[width=0.95\textwidth]}{../img/mesh2.png}
\caption{Mesh of the domain 2.}
\label{reflection_mesh2}
\end{figure}

We can see on figure \ref{reflection_result2} that the results are simply rotated from those of figure \ref{reflection_result}. 

\begin{figure} [!h]
\centering
\includegraphicsmaybe{[width=0.95\textwidth]}{../img/results2.png}
\caption{Wave height hm0.}
\label{reflection_result2}
\end{figure}

Plotting the wave height on point [0,0] of first geometry and [0,0] of second geometry, we can see that it is the same result.

\begin{figure} [!h]
\centering
\includegraphicsmaybe{[width=0.8\textwidth]}{../img/waveheight.png}
\caption{Wave height on point [0,0].}
\label{result_point0}
\end{figure}
