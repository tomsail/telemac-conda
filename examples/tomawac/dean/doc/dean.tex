\chapter{Dean}
%

% - Purpose & Problem description:
%     These first two parts give reader short details about the test case,
%     the physical phenomena involved and specify how the numerical solution
%     will be validated
%
\section{Purpose}
%
This test case aims to verify the developpement made to take into account the
wave energy dissipation induced by vegetation. When the ratio between
vegetation height and water depth is important, dissipation may become non
negligible.
%
\section{Description of the problem}
%
In order to carry out an analysis of the influence of plant height, vegetation
field width and breaking on waves propagation, Mendez and Losada (2004)
analysed the evolution of the wave height over a Dean’s shape profile
\cite{Dean1991} defined as follows:
$$
h=0.25(300-x)^\frac{2}{3}
$$
Where h [m] is the water depth, 0.25 the sediment scale parameter, and x=0 is
the offshore boundary.

\section{Reference}
%
In this test case the prediction of the effects of vegetation is validated
with the original equation and results from Mendez and Losada
\cite{Mendez2004}.The reference file {\it fom\_dean.slf} contains the results
that have been compared to \cite{Mendez2004} in \cite{Bacchi2014}.

\section{Physical parameters}
%
Two vegetation heights, dv = 1~m and 3~m and a single 100~m long vegetation
field, from 50 to 150~m, are used. The number of plants per square meter is
$N = 20~units/m^2$ and the plant area per unit height of vegetation is
$b_v = 0.25~m.$ The bulk drag coefficient is 0.2. All these parameters are
keywords in the steering file.


\section{Geometry and Mesh}
%
The computational domain was composed of a flat (slope = 0.0) 2D grid with an
aspect ratio of 1 (cross-shore direction):10 (along shore direction). The
calculation grid size was set as 2.0 m in the wave propagation direction.
Bathymetry and mesh are shown on figure \ref{bathydean}. There are 906 nodes
and 1500 triangles.
\begin{figure} [!h]
\centering
\includegraphicsmaybe{[width=0.85\textwidth]}{../img/mesh.png}
%\includegraphics[width=0.85\textwidth]{bathy.png}
 \caption{Depth evolution.}
\label{bathydean}
\end{figure}

The area where vegetation is taken into account is defined by the file
{\it Zone.txt'}

\section{Initial and Boundary Conditions}
%
According to the authors, the incident wave conditions imposed to \tomawac
on the offshore boundary are given by $H_{rms,o} = 2.5~m$ (equivalent to
significant wave height $H_s = 3.54~m$) and $T_p = 10~s$. The incident waves
are uni-directional random waves as defined in the previous section and the
breaking model used is that of Thornton and Guza (1983) with  $\gamma=0.6$
(where the parameter $\gamma$  is the proportional control factor indicating
the maximum water depth “Hm” compatible with water depth “d”: ).

The significant initial wave heigth was taken equal to 3.54~m with a peak
frequency of 2.2~Hz. The angular distribution function follows a $\cos^{2s}
\theta$ distribution with an angular spreading of 2, and a mean direction of 90.

\section{Numerical parameters}
%
The time step is of 0.1~s and the duration of the computation of 480~s. The
spectro-angular mesh has 36 directions and 6 frequencies. The frequential ratio
is of 1.01 and the minimum frequency of 0.0951~Hz. The frequencies are filtered
to keep only the fourth one (0.1 Hz) at 90 degrees to respect the $T_p= 10~s$
imposed to the boundaries.

The spectrum tail factor was taken at 4.

\section{Results}
%
The results from Mendez and Losada and \tomawac model are compared in Fig.
\ref{figresvito} below. Even if the test case is made with 1~m of vegetation,
we present on Figure \ref{figresvito} the results obtained for three different
heigth of vegetation, 0~m, 1~m and 3~m \cite{Bacchi2014}.
The results show very good agreement between the Mendez  and Losada model
\cite{Mendez2004} and \tomawac. We notice that differences seem very small and
we can thus conclude that \tomawac is able to reproduce the same wave
attenuation as with the random wave transformation model for breaking
uni-directional random waves.
\begin{figure} [!h]
\centering
\includegraphics[width=0.85\textwidth]{resdean.png}
\caption{Comparison of Hrms evolution for numerical wave model (\tomawac) and
  random wave transformation model (Mendez and Losada) over Dean’s shape
  profile.}
\label{figresvito}
\end{figure}

\begin{figure} [!h]
\centering
\includegraphicsmaybe{[width=0.85\textwidth]}{../img/section1d.png}
 \caption{Wave height for dv=1m.}
\label{figrescalc}
\end{figure}

\begin{figure} [!h]
\centering
\includegraphicsmaybe{[width=0.85\textwidth]}{../img/kmoyen.png}
 \caption{Average Wave number for dv=1m.}
\label{figkmoy}
\end{figure}


