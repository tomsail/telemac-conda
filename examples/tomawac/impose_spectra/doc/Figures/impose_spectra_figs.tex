\documentclass[oneside,10pt]{article}

%%%%%%%%%%%%%
%Mise en page
%%%%%%%%%%%%%
%\usepackage[margin=0cm,papersize={5.5cm,5cm}]{geometry}
%%max: \usepackage[margin=0cm,papersize={16cm,24cm}]{geometry}
\usepackage{nopageno}

%---------------------------------------------------------------------------
%Equations
%---------------------------------------------------------------------------
\usepackage{amsfonts,amsmath,amssymb,mathtools}
\usepackage{bm} %pour avoir des symbol en gras quand ils ne sont pas défini dans mathbf
\usepackage{amsbsy}
%\usepackage{xargs}
\renewcommand{\vec}[1]{\bm{#1}}
%\newcommand{\ra}[1]{\renewcommand{\arraystretch}{#1}}
%% You can define your own commands like this one for the average flow velocity notation:
%\newcommand{\afv}[1]{\mathcal{h}#1\mathcal{i}}
% or to have sums in dispaystyle:
\newcommand{\dsum} {\displaystyle\sum}
\setcounter{MaxMatrixCols}{10}

\newcommand{\Grad}{\text{\bfseries{Grad}}}
\newcommand{\Div}{\text{Div}}
\newcommand{\Lap}{\text{Lap}}


%%%%%%%%%%%%%
%Pour les graphiques
%%%%%%%%%%%%%
\usepackage{float}
\usepackage{tikz}
\usetikzlibrary{arrows,external,calc,positioning,intersections}
%\tikzexternalize[prefix=tikz/,only named=true]
\tikzexternalize[only named=true]
% rajouter " -shell-escape"  la commande pdflatex ou xelatex
% l'option "only named=true" fait en sorte que seulement les figures avec un nom sont externalises
% Pour definir le nom de l'image :
% \tikzsetnextfilename{<file_name>}}

%\usetikzlibrary{arrows,calc,positioning,intersections}
%%%\tikzexternalize[prefix=tikz/,only named=true]
%%\tikzexternalize[only named=true]
%%% rajouter " -shell-escape"  la commande pdflatex ou xelatex
%%% l'option "only named=true" fait en sorte que seulement les figures avec un nom sont externalises
%%% Pour definir le nom de l'image :
%%% \tikzsetnextfilename{<file_name>}}




% pour avoir dash-dot
  \tikzset{
    dash-dot/.style={
    dash pattern=on 4pt off 3pt on 1pt off 3pt,
  },
  }
% pour avoir dash-dot-dot
  \tikzset{
    dash-dot-dot/.style={
    dash pattern=on 4pt off 3pt on 1pt off 3pt on 1pt off 3pt,
  },
  }

\usepackage{pgfplots}
\usetikzlibrary{pgfplots.groupplots}
	% Pour que ce soit compatible avec la version 1.6
	\pgfplotsset{compat=1.6}
	% pour les label des axes
	\pgfplotsset{tick scale binop={\times}}
%	% le style de la légende  
%	\pgfplotsset{every axis legend/.append style={
%		at={(0.9,0.9)},
%		anchor=north east}}
%	% pour avoir un espace separant les mille
	\pgfkeys{/pgf/number format/.cd,fixed,
		precision=4,
		%use comma, %pour utiliser la virgule
		set thousands separator={\,}}
%	% position des label sur x et y
%	\pgfplotsset{every axis x label/.style=
%		{at={(ticklabel cs:0.5)},anchor=near ticklabel}}
%	\pgfplotsset{every axis y label/.style=
%		{at={(ticklabel cs:0.5)},anchor=near ticklabel}}
%	% Modifier les marks
%	\pgfplotsset{every mark/.append style={solid}}
	% pour augmenter la taille des lignes
	\pgfplotsset{every axis/.append style={
		line width=1pt,
		axis line style={line width=0.4pt},
		tick style={line width=0.4pt}}}



\definecolor{LHSVLightBlue}{cmyk}{0.99,0.29,0.00,0.12}
\definecolor{LHSVDarkBlue}{cmyk}{1.00,0.35,0.00,0.20}
\definecolor{EdfBlue}{RGB}{0,91,187}
\definecolor{EdfDBlue}{RGB}{9,53,122}
\definecolor{EdfLBlue}{RGB}{56,174,255}
\definecolor{EdfOrange}{RGB}{255,160,47}
\definecolor{EdfLOrange}{RGB}{243,157,0}
\definecolor{EdfGreen}{RGB}{80,158,47}
\definecolor{EdfDGreen}{RGB}{196,214,47}
\definecolor{PantoneRed}{RGB}{232,17,45}
\definecolor{PantonePurple}{RGB}{137,79,191}
\definecolor{PantoneGreen}{RGB}{0,153,124}
\definecolor{PantonePink}{RGB}{214,2,112}

\begin{document}

%------------------------------------
\section{Mesh Illustration}
%------------------------------------

\tikzset{external/force remake=true}
\tikzsetnextfilename{MeshIllustration}

\begin{figure}[H]
\begin{center}
%
		\begin{tikzpicture}[baseline]
		\begin{groupplot}[
    		group style={
        	group name=MyPlots,
        	group size=1 by 2,
        	%xlabels at=edge bottom,
        	%xticklabels at=edge bottom,
			%x descriptions at=edge bottom,
        	vertical sep=10pt
    		},
%% légende
%		%legend pos=east,
%		legend style={at={(1,0.5)},anchor=east},
% Pour avoir les axes en dessus
		enlargelimits=false,axis on top=true,
% titres des axes
		ylabel=$y$ (m),
		%xlabel=$\dfrac{x}{L}$,
% Option des axes
%  		axis x line=none,
  		%xticklabels={,,},
%		axis x line=bottom,axis y line=left,
%		every outer y axis line/.append style={<-},
%		y dir=reverse,
% largeur du graphiques
		width=\linewidth,height=0.5\linewidth,%scale only axis,
% max et min
		xmin=-200,xmax=200,ymin=0,ymax=200
		]
		\nextgroupplot[xticklabels={,,}]
			\addplot graphics [xmin=-200,xmax=200,ymin=0,ymax=200] {./OceanicMeshNoAxis.png};
		%
		\nextgroupplot[xlabel=$x$ (m)]
			\addplot graphics [xmin=-200,xmax=200,ymin=0,ymax=200] {./CoastalMeshNoAxis.png};
		%
		\end{groupplot}
		% Legend
		\draw[color=EdfBlue,line width=3pt] (MyPlots c1r2.south west) ++ (0.05\linewidth,-1.5cm) -- ++ (1cm,0) node[anchor=west] {Oceanic Mesh};
		\draw[color=PantoneRed,line width=3pt] (MyPlots c1r2.south west) ++ (0.55\linewidth,-1.5cm) -- ++ (1cm,0) node[anchor=west] {Coastal Mesh};
		\end{tikzpicture}
%
\end{center}
\end{figure}

\end{document}