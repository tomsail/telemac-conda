\chapter{3D coupling}
%
% - Purpose & Problem description:
%     These first two parts give reader short details about the test case,
%     the physical phenomena involved and specify how the numerical solution will be validated
%
\section{Purpose}
%
This test case has been created to test the coupling between \tomawac and
\telemac3. Four different coupling are tested here:
\begin{itemize}
\item In the first one that we will call the classical one, the forces due to
  waves are constant along the depth of water. This coupling is very close to
  the coupling that is made with telemac2d.
\item In the second coupling that we will call 3D coupling, all the
quantities are dependant of the depth. This coupling is closer to what
happen in reality. For more detail about the second coupling the reader can
read \cite{Teles2013}.
\item The third coupling is like the first one but \tomawac and \telemac3 are
  defined on different domains and meshes using TEL2TOM technique
  (see \cite{breugem2019} for a detailed description). The mesh of \tomawac is twice coarser and a little bit larger. 
\item The forth coupling is a 3D coupling where \tomawac and \telemac3 are
  defined on different domains and meshes.
\end{itemize}
%
\section{Description of the problem}
We took the geometry from the classical test case 'littoral', a coupling case
between \tomawac and \telemac2 and Sisyphe
\section{Geometry and Mesh}
%
The beach is 1000 m long, 200 m wide with a regular mesh with
elements size of the order $\Delta x=20$~m and $\Delta y=5$~m
 The beach slope (Y=200m) is 5\%.
 The water depth along the open boundary (Y=0) is h=10m
We use a trianglular regular grid.  
The mesh is as shown on Figure \ref{3Dcouplingmesh}


\begin{figure} [!h]
\centering
\includegraphicsmaybe{[width=0.85\textwidth]}{../img/fond.png}
 \caption{2D mesh of the domain in the two first coupling.}
\label{3Dcouplingmesh}
\end{figure}

\begin{figure} [!h]
\centering
\includegraphicsmaybe{[width=0.85\textwidth]}{../img/mesh5T3D.png}
\includegraphicsmaybe{[width=0.85\textwidth]}{../img/mesh5WAC.png}
 \caption{2D mesh of the domain in the two last coupling. \telemac3 up and \tomawac bottom}
\label{3Dcouplingmesh2}
\end{figure}


\subsection{Wave conditions}
Incoming waves (waves height , period and directions) are imposed offshore at
$y=0$, such that $H_s=1$~m, $T_p=8$~s. The Jonswap spectrum is used. The waves
direction is $30$~deg relative to the $y-$axis.

$\Rightarrow $ Offshore (Y=0): Offshore wave imposed/no littoral current/no
set up

Tomawac:
The wave height is imposed on the offshore boundary (5 4 4) (Hs=1m), for a
wave period (Tp=8s).

Telemac2D:
The current and free surface are imposed to 0 along the offshore boundary
(5 5 5).

\section{Results}
The results are presented Figures \ref{figres3Dcoupl2} (Velocity U on a
vertical plan) \ref{figres3Dcoupl3} (velocity on the bottom)
\ref{figres3Dcoupl}(Wave heigth Hm0) with the classical coupling.

On Figures \ref{figres3Dcoupl4}, \ref{figres3Dcoupl5} and \ref{figres3Dcoupl6},
we present the results of the 3D coupling. 

\begin{figure} [!h]
\centering
\includegraphicsmaybe{[width=0.85\textwidth]}{../img/HM01.png}
 \caption{Wave Heigth calculated by \tomawac.}
\label{figres3Dcoupl}
\end{figure}

\begin{figure} [!h]
\centering
\includegraphicsmaybe{[width=0.85\textwidth]}{../img/resultscoupVert.png}
 \caption{Celerity U on a vertical plan at x=500.}
\label{figres3Dcoupl2}
\end{figure}

\begin{figure} [!h]
\centering
\includegraphicsmaybe{[width=0.85\textwidth]}{../img/resultshori.png}
\caption{Celerity U on the bottom calculated by \telemac3d.}
\label{figres3Dcoupl3}
\end{figure}

\begin{figure} [!h]
\centering
\includegraphicsmaybe{[width=0.85\textwidth]}{../img/HM03.png}
\caption{Wave Heigth calculated by \tomawac with the 3D coupling.}
\label{figres3Dcoupl4}
\end{figure}

\begin{figure} [!h]
\centering
\includegraphicsmaybe{[width=0.85\textwidth]}{../img/resultscoupVert3.png}
\caption{Celerity U on a vertical plan at x=500.}
\label{figres3Dcoupl5}
\end{figure}

\begin{figure} [!h]
\centering
\includegraphicsmaybe{[width=0.85\textwidth]}{../img/resultshori3.png}
\caption{Celerity U on the bottom calculated by \telemac3d with the 3D
  coupling.}
\label{figres3Dcoupl6}
\end{figure}


\begin{figure} [!h]
\centering
\includegraphicsmaybe{[width=0.85\textwidth]}{../img/HM05.png}
 \caption{Wave Heigth calculated by \tomawac with a claccical coupling on different meshes }
\label{figres3Dcoupl7}
\end{figure}

\begin{figure} [!h]
\centering
\includegraphicsmaybe{[width=0.85\textwidth]}{../img/HM07.png}
 \caption{Wave Heigth calculated by \tomawac with a real coupling coupling on different meshes }
\label{figres3Dcoupl8}
\end{figure}
