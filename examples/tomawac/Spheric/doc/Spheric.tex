\chapter{Spheric}
%
% - Purpose & Problem description:
%     These first two parts give reader short details about the test case,
%     the physical phenomena involved and specify how the numerical solution will be validated
%
\section{Purpose}
%
This test case has been made to test spherical coordinate. 

\section{Description of the problem}

The case look like channel sea but we intentionnaly put an infinite depth. 

We discretize the spectrum with 31 frequencies and 36 directions. The minimal
frequency is 0.04 and the frequential ratio 1.1.

The time step is 180s and we simulate during 50mn. 

We take into account the wind through a file that gives the wind in all point
of the mesh every 36 hours, so in our case we can consider constant wind (see
Figure \ref{figSphericVent}). As for current, it is taken into account through
a fileand is update every 1800s so every 10 time steps(see Figure
\ref{figSphericCourrant}).

\begin{figure} [!h]
\centering
\includegraphicsmaybe{[width=0.85\textwidth]}{../img/mesh.png}
 \caption{Mesh of the domain.}
\label{meshSpheric}
\end{figure}

\begin{figure} [!h]
\centering
\includegraphicsmaybe{[width=0.85\textwidth]}{../img/vent.png}
 \caption{Wind.}
\label{figSphericVent}
\end{figure}
\begin{figure} [!h]
\centering
\includegraphicsmaybe{[width=0.85\textwidth]}{../img/courant.png}
 \caption{Current.}
\label{figSphericCourrant}
\end{figure}

\section{Results}
We present Figure \ref{figSphericHm0} the wave height obtained after 50mn. 
\begin{figure} [!h]
\centering
\includegraphicsmaybe{[width=0.85\textwidth]}{../img/hm0.png}
 \caption{Wave height hm0.}
\label{figSphericHm0}
\end{figure}
