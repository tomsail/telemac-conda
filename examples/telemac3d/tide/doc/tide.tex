\chapter{Propagation of tide prescribed by boundary conditions (tide)}

\section{Purpose}

This test demonstrates the availability of \telemac{3D} to model the
propagation of tide in a maritime domain by computing tidal
boundary conditions.
A coastal area located in the English Channel off the coast of
Brittany (in France) close to the real location of the Paimpol-Bréhat
tidal farm is modelled to simulate the tide and the tidal currents
over this area.
Time and space varying boundary conditions are prescribed over
liquid boundaries.

\section{Description}

\subsection{Geometry and Mesh}

The geometry of the domain is almost a rectangle with the French coasts on
one side (22~km $\times$ 24~km).
The triangular mesh is composed of 4,385 triangular elements and 2,386 nodes
(see Figure \ref{fig:tide:mesh}).

\begin{figure}[H]
 \centering
 \includegraphicsmaybe{[width=0.6\textwidth]}{../img/geom_mesh.png}
  \caption{Horizontal mesh.}\label{fig:tide:mesh}
\end{figure}

To build the 3D mesh of prisms, 11 planes are regularly spaced over the vertical.
A slice of 3D mesh between nodes of coordinates (185,500 ; 150,000) to
(200,500 ; 150,000) can be seen in Figure \ref{fig:tide:mesh:section}.

\begin{figure}[H]
 \centering
 \includegraphicsmaybe{[width=0.8\textwidth]}{../img/res_mesh_section.png}
  \caption{Slice of 3D mesh.}\label{fig:tide:mesh:section}
\end{figure}

\subsection{Bathymetry}

A real bathymetry of the area bought from the SHOM (French Navy
Hydrographic and Oceanographic Service) is used.
\copyright Copyright 2007 SHOM. Produced with the permission of SHOM.
Contract number 67/2007

\subsection{Initial conditions}

Initial conditions are defined by a constant elevation and no velocity.

\subsection{Boundary conditions}

Several databases of harmonic constants are interfaced with
\telemac{3D}:
\begin{itemize}
\item The JMJ database resulting from the LNH Atlantic coast TELEMAC
model by Jean-Marc JANIN,
\item The global TPXO database and its regional and local variants
from the Oregon State University (OSU),
\item The regional North-East Atlantic atlas (NEA) and the global
atlas FES (e.g. FES2004 or FES2012...) coming from the works of
Laboratoire d’Etudes en Géophysique et Océanographie Spatiales (LEGOS),
\item The PREVIMER atlases.
\end{itemize}

In the tide test case, only the JMJ database is used as example.
The user can read the \telemac{2D} validation document and the \telemac{2D}
examples to see how it is done with NEA atlas or TPXO-like tidal solutions,
and of course the \telemac{3D} or \telemac{2D} user manuals.
Elevation and horizontal velocity boundary conditions are computed by
\telemac{3D} from an harmonic constants database (JMJ from LNH).
If a tidal solution from OSU has been downloaded (e.g. TPXO, European
Shelf), it can be used to compute elevation and horizontal velocity
boundary conditions as well.

\subsection{Physical parameters}

Vertical turbulence model: mixing length model\\
Horizontal viscosity for velocity: $10^{-4}~\rm{m}^2$/s\\
Coriolis: yes (constant coefficient over the domain
= 1.10 $\times$ 10$^{-4}$~rad/s)\\
No wind, no atmospheric pressure, no surge and nor waves

\subsection{Numerical parameters}

Time step: 20~s\\
Simulation duration: 90,000~s = 25~h\\

Non-hydrostatic version\\
Advection for velocities: Characteristics method\\
Thompson method with calculation of characteristics for open boundary
conditions\\
Free Surface Gradient Compatibility = 0.5 (not 0.9) to prevent on
wiggles\\
Tidal flats with correction of Free Surface by elements, treatments
to have $h \ge 0$

\subsection{Comments}

If a tidal solution from OSU has been downloaded (e.g. TPXO, European
Shelf), it can be used to compute initial conditions with the keyword
\telkey{INITIAL CONDITIONS} set to TPXO SATELLITE ALTIMETRY.
Thus, both initial water levels and horizontal components of velocity
can be calculated and may vary in space.

\section{Results}

Tidal range, sea levels and tidal velocities are well reproduced compared to
data coming from the SHOM or at sea measurements.
In Figure \ref{fig:tide:H} the water depth and free surface elevation at final
time are shown in the case of the JMJ database.
In Figure \ref{fig:tide:U} the surface velocity magnitude, vectors and
streamlines are plotted from the \telkey{3D RESULT FILE}.
If using the \telkey{2D RESULT FILE}, a mask for tidal flats could be added
for variable fields.

\begin{figure}[!htbp]
\begin{minipage}[t]{0.5\textwidth}
 \centering
 \includegraphicsmaybe{[width=\textwidth]}{../img/r2d_water_depth.png}
\end{minipage}%
\begin{minipage}[t]{0.5\textwidth}
 \centering
% \includegraphicsmaybe{[width=\textwidth]}{../img/r2d_elevation.png}
 \includegraphicsmaybe{[width=\textwidth]}{../img/res_z_map.png}
\end{minipage}
  \caption{Water depth (in m) and free surface elevation (in m CD) at final time.}\label{fig:tide:H}
\end{figure}

\begin{figure}[H]
\begin{minipage}[t]{0.5\textwidth}
 \centering
 \includegraphicsmaybe{[width=\textwidth]}{../img/res_velocity_surf_vectors.png}
\end{minipage}%
\begin{minipage}[t]{0.5\textwidth}
 \centering
 \includegraphicsmaybe{[width=\textwidth]}{../img/res_velocity_surf_streamlines.png}
\end{minipage}
  \caption{Surface velocity (in m/s) at final time (with vectors on the left and streamlines on the right).}
  \label{fig:tide:U}
\end{figure}

Depth-averaged velocity magnitude, vectors and streamlines can be plotted
from the \telkey{2D RESULT FILE} (see Figure \ref{fig:tide:average:U}).

\begin{figure}[H]
\begin{minipage}[t]{0.5\textwidth}
 \centering
 \includegraphicsmaybe{[width=\textwidth]}{../img/r2d_velocity_vectors.png}
\end{minipage}%
\begin{minipage}[t]{0.5\textwidth}
 \centering
 \includegraphicsmaybe{[width=\textwidth]}{../img/r2d_velocity_streamlines.png}
\end{minipage}
  \caption{Depth-averaged velocity (in m/s) at final time (with vectors on the left and streamlines on the right).}
  \label{fig:tide:average:U}
\end{figure}

Surface velocity can also be plotted from the \telkey{2D RESULT FILE} with a
mask for tidal flats (see Figure \ref{fig:tide:surf:U}).

\begin{figure}[H]
 \centering
 \includegraphicsmaybe{[width=0.6\textwidth]}{../img/r2d_surface_velocity.png}
  \caption{Surface velocity (in m/s) at final time with mask for tidal flats.}\label{fig:tide:surf:U}
\end{figure}

\section{Conclusion}

\telemac{3D} is able to model tide in coastal areas.
