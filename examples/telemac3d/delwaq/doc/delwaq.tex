\chapter{One way chaining with DELWAQ (delwaq)}

\section{Description}

This test demonstrates the availability of \telemac{3D} to be chained
with DELWAQ, the water quality software from Deltares.
This is a one way-chaining by files.

A 20~m wide and 28.5~m long prismatic channel with trapezoidal cross-section
contains bridge-like obstacles in one cross-section made of two abutments and two
circular 4~m diameter piles (see Figure \ref{t3d:delwaq:fig:Bottom}).

\begin{figure}[!htbp]
 \centering
 \includegraphicsmaybe{[width=0.9\textwidth]}{../img/Bottom.png}
 \caption{Bottom elevation.}
 \label{t3d:delwaq:fig:Bottom}
\end{figure}

The flow resulting from steady state boundary conditions is studied.
The deepest water depth is 4~m.
The hydrodynamic part is similar to the pildepon test case.
The tracer used is temperature.

\subsection{Initial and boundary conditions}

The computation is initialised with a constant elevation equal to 0~m,
no velocity and a uniform temperature at 0.

The boundary conditions are:
\begin{itemize}
\item For the solid walls, a slip condition on channel banks is used for the
velocities,
\item On the bottom, a Strickler law with friction coefficient equal to
40~m$^{1/3}$/s is prescribed
and imposed uniform profile of temperature over the vertical along
a segment equal to 1,
\item Upstream a flowrate equal to 62~m$^3$/s is prescribed,
linearly increasing from 0 to 62~m$^3$/s during the first 10~s,
\item Downstream the water level is equal to 0~m.
\end{itemize}

\subsection{Mesh and numerical parameters}

The 2D mesh (Figure \ref{t3d:delwaq:fig:meshH})
is made of 4,304 triangular elements (2,280 nodes).
6 planes are regularly spaced on the vertical (see Figure \ref{t3d:delwaq:fig:meshV}).

\begin{figure}[!htbp]
 \centering
 \includegraphicsmaybe{[width=0.9\textwidth]}{../img/MeshH.png}
 \caption{Horizontal mesh.}
 \label{t3d:delwaq:fig:meshH}
\end{figure}

\begin{figure}[!htbp]
 \centering
 \includegraphicsmaybe{[width=0.9\textwidth]}{../img/MeshV.png}
 \caption{Initial vertical mesh.}
 \label{t3d:delwaq:fig:meshV}
\end{figure}

The time step is 0.1~s for a simulated period of 5~s.

The non-hydrostatic version is used.
To solve the advection, the LIPS scheme (default value)
is used for the velocities and tracers (scheme 5 + scheme option 4).
The conjugate gradient
is used for solving the propagation step (option 1) and
the implicitation coefficients
for depth and velocities are respectively equal to 0.6 and 0.55
(= default value for velocity).

\subsection{Physical parameters}

A mixing length model is used to model turbulence over the vertical
while a constant horizontal viscosity for velocity equal to 0.005~m$^2$/s is
chosen.

\section{Results}

Figure \ref{t3d:delwaq:FreeSurf} shows the free surface elevation at the end of
the computation.

\begin{figure}[H]
  \centering
  \includegraphicsmaybe{[width=0.9\textwidth]}{../img/FreeSurface.png}
  \caption{Free surface at final time step.}
  \label{t3d:delwaq:FreeSurf}
\end{figure}

Figure \ref{t3d:delwaq:Velo} shows the magnitude of velocity at the end of the
computation.

\begin{figure}[H]
  \centering
  \includegraphicsmaybe{[width=0.9\textwidth]}{../img/VelocityH.png}
  \caption{Velocity magnitude at the surface at final time step.}
  \label{t3d:delwaq:Velo}
\end{figure}

If running DELWAQ with the DELWAQ result files written by \telemac{3D}, the
velocity results are similar with both codes.

\section{Conclusion}

\telemac{3D} can be used to chain with DELWAQ.
