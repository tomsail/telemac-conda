\chapter{Flow in a channel with bottom friction (canal)}

\section{Description}
\bigskip
This study case verifies that \telemac{3D} is able to compute the free
surface evolution along a channel with bottom friction.

This example also checks that \telemac{3D} is able to deal with stage-discharge
curves.

\bigskip
The chosen configuration is a straight channel 500~m long and 100~m wide
with a flat horizontal bottom without slope.
Four different cases are studied:
\begin{itemize}
\item A 2D computation using \telemac{2D},
\item A 3D computation with hydrostatic option,
\item A 3D computation with non-hydrostatic option,
\item A 3D computation with non-hydrostatic option and
a stage-discharge curve $Z = f(Q)$ at the exit.
\end{itemize}
In all cases, the flow establishes a steady flow where the free surface
is influenced by the friction on the bottom. \\
Note that the velocity diffusion is constant in horizontal direction
and equal to $0.1~\text{m}^2.\text{s}^{-1}$ with the non-hydrostatic 
option and with \telemac{2D} usage, whereas with the hydrostatic option, 
it is null. In the vertical direction, the turbulent viscosity uses 
Nezu and Nakagawa mixing length model for the three \telemac{3D} computations.

\subsection{Initial and boundary conditions}

\bigskip
This test is initialised by a steady flow (3D cases initialised 
from the 2D case result file, Figures 
\ref{t3d:canal:fig:veloH} and \ref{t3d:canal:fig:freeSurface}).

\bigskip
The boundary conditions are:
\begin{itemize}
\item For the solid walls, a slip condition on channel banks is used for the velocities,
\item On the bottom, Strickler law with friction coefficient equal to
50~$\text{m}^{1/3}.\text{s}^{-1}$ is imposed,
\item Upstream a flowrate equal to 50~$\text{m}^{3}.\text{s}^{-1}$ is imposed,
\item Downstream the water level is equal to 0.5~m for the first three
computations.
For the last computation, elevation is also prescribed but by giving a
stage-discharge curve $Z = f(Q)$.
\end{itemize}
\begin{figure}[!htbp]
 \centering
 \includegraphicsmaybe{[width=\textwidth]}{../img/veloH.png}
 \caption{Horizontal velocity obtained at the steady state.}
 \label{t3d:canal:fig:veloH}
\end{figure}

\begin{figure}[!htbp]
 \centering
 \includegraphicsmaybe{[width=\textwidth]}{../img/freeSurfac.png}
 \caption{Free surface at the steady state.}
 \label{t3d:canal:fig:freeSurface}
\end{figure}

\subsection{Mesh and numerical parameters}
\bigskip
The mesh (Figure \ref{t3d:canal:fig:meshH} and \ref{t3d:canal:fig:meshV})  
is composed of 551 triangular elements (319 nodes) with 10 planes  
regularly spaced on the vertical, to form prism elements.\\

\begin{figure}[!htbp]
 \centering
 \includegraphicsmaybe{[width=\textwidth]}{../img/Mesh.png}
 \caption{Horizontal mesh.}
 \label{t3d:canal:fig:meshH}
\end{figure}
\begin{figure}[!htbp]
 \centering
 \includegraphicsmaybe{[width=\textwidth]}{../img/MeshV.png}
 \caption{Vertical mesh.}
 \label{t3d:canal:fig:meshV}
\end{figure}


\bigskip
The time step is 2~s for a simulated period of 4,000~s for the 2D computation
then 2,000~s more for 3D computations.

\bigskip
This case is computed both with the hydrostatic pressure assumption and the non-hydrostatic version. 
To solve the advection, the method of characteristics
is used for the velocities (scheme 1). The GMRES 
(Generalized Minimal Residual Method, scheme 7) is used to solve 
the vertical velocity.  The implicitation coefficients 
for depth and velocities are equal to 0.6.

For the 2D resolution, the conjugate gradient
is used for solving the propagation step (option 1) and
the implicitation coefficients 
for depth and velocities are respectively equal to 1 and 0.55. 

\section{Results}

\bigskip
In Figure \ref{t3d:canal:fig:profiles}, the three free surface profiles
corresponding to the first three simulations are compared.
These three results are in good agreement.
We can also observe that the flow is completely symmetric without
any influence of the space discretisation.

\begin{figure}[!htbp]
 \centering
 \includegraphicsmaybe{[width=0.9\textwidth]}{../img/free_surface.png}
 \caption{Comparison of the free surface profiles between the 2D 
 computation  and the 3D computations 
 with hydrostatic and non-hydrostatic options. }
 \label{t3d:canal:fig:profiles}
 \end{figure}

The same comments can be done when comparing classical prescribed boundary
conditions and boundary conditions with stage-discharge curve,
see Figure \ref{t3d:canal:fig:profiles:sta-dis-cur}.

\begin{figure}[!htbp]
 \centering
 \includegraphicsmaybe{[width=0.9\textwidth]}{../img/free_surface_sta_dis.png}
 \caption{Comparison of the free surface profiles between the computations
 without or with stage-discharge curves.}
 \label{t3d:canal:fig:profiles:sta-dis-cur}
\end{figure}

\section{Conclusion}

To conclude, \telemac{3D} is able to take into account correctly the bottom
friction term and stage-discharge curves.
