\chapter{Sources of fluid and tracers (source)}

\section{Description}
\bigskip
%This test is a basin at rest with source points 
%injecting tracers and fluid.
This test shows the capability of \telemac{3D} to manage multiple
sources of fluid and tracers.
It also demonstrates the ability to compute injection and conservation
of multiple tracers.

\bigskip
A 100~m long and 40~m wide flat bottom basin with a constant water depth of 1~m is
considered (the bottom being $z$ = -1~m). The fluid is at rest.
Sources are specified at two points of the mesh.
Three tracers are used, one is the sum of the two others (tracer 2).\\
Note that the turbulent viscosity is constant in both directions 
and equal to the water molecular viscosity (10$^{-6}$ m$^2\cdot\text{s}^{-1}$,
default value).

\subsection{Initial and boundary conditions}

\bigskip
Initially, the fluid is at rest and the water level is null.

\bigskip
The boundary conditions are:
\begin{itemize}
\item Solid boundaries without roughness (slip conditions) everywhere,
 i.e., no entrance and no outlet in the domain,
\item On the bottom, Chézy law with friction coefficient equal to 
60~$\text{m}^{1/2}.\text{s}^{-1}$ is imposed (default value
for the friction coefficient until version 8.0).
\end{itemize}

%\subsection{Definition of sources}
\bigskip
The sources are defined at following positions:
\begin{itemize}
\item Position source 1: $x \approx$ -21.6~m, $y \approx$ 5.3~m, $z$ = -0.5~m,
\item Position source 2: $x \approx$ -0.8~m, $y \approx$ -10.~m, $z$ = -0.5~m
\end{itemize}
A constant discharge of 1~m$^3$.s$^{-1}$ is imposed at both sources.
The source 1 diffuses without any velocity in every direction , 
whereas source 2 has an initial velocity: 
$U$ = 0.5~m.s$^{-1}$, $V$ = 2~m.s$^{-1}$.
At source 1, the tracers 1 and 2 are discharged. At source 2, 
the tracers 2 and 3 are discharged with a concentration 
of 10~kg.m$^{-3}$ (or g.L$^{-1}$) for all tracers.\\
Note that the definition of the discharges and tracer concentrations at sources
has been done in the \telkey{SOURCES FILE}.
As this last file is present in the examples directory, the keyword
\telkey{VALUES OF THE TRACERS AT THE SOURCES} is ignored but is given
as an example coherent with what is done in the \telkey{SOURCES FILE}.

\subsection{Mesh and numerical parameters}

\bigskip
Figure \ref{t3d:source:fig:meshH} shows the horizontal mesh and sources
positions. The mesh is composed of 674 triangular elements (373 nodes) with 5 planes  
regularly spaced on the vertical, to form prism elements.\\

\begin{figure}[!htbp]
 \centering
 \includegraphicsmaybe{[width=\textwidth]}{../img/Mesh.png}
 \caption{Horizontal mesh.}
 \label{t3d:source:fig:meshH}
\end{figure}

The time step is 1.1~s for a simulated period of 1,100~s.

\bigskip
This case is computed with the hydrostatic pressure assumption. 
To solve the advection, the method of characteristics 
is used for the velocities (scheme 1, default until version 8.0)
and the explicit finite volume 
Leo Postma scheme for tidal flats is used for the tracer (scheme 13).


\section{Results}

\begin{figure}[!htbp]
\centering
\includegraphicsmaybe{[width=\textwidth]}{../img/velo_vec.png}
 \caption{Horizontal velocity field at mid depth at 550~s.}
 \label{t3d:source:velo}
\end{figure}

\bigskip
Figure \ref{t3d:source:velo} highlights the influence of the initial velocity of
source 2 on the horizontal velocity field at mid depth at 550~s.
Additionally, it shows tracer 2 spreads in every direction
at source 1, unlike at source 2 where tracer 2 diffuses in the initial
velocity direction.\\
The horizontal and vertical plumes of each tracer at 1,100~s, in
Figures \ref{t3d:source:hor_shape} and \ref{t3d:source:ver_shape}
respectively, allow verifying that the plume of tracer 2 is the
combination of the plumes of tracer 1 and 3.

\begin{figure}[!htbp]
\centering
\includegraphicsmaybe{[width=\textwidth]}{../img/plumTr_H.png}
 \caption{Horizontal shape of the plumes for each tracers.}
 \label{t3d:source:hor_shape}
\end{figure}

\begin{figure}[!htbp]
\centering
\includegraphicsmaybe{[width=\textwidth]}{../img/plumTr_V.png}\\
 \caption{Vertical shape of the plumes for each tracers.}
 \label{t3d:source:ver_shape}
\end{figure}
\bigskip
Moreover, the following mass balance of the \telemac{3D} simulation
shows that the amount of water injected by the sources is very good
(2,200~m$^3$ = 2 sources $\times$ 1~m$^3\cdot\text{s}^{-1}~\times$ 1,100~s
with an error lower than 10$^{-6}~\rm{m}^3$, thus a relative error
lower than 10$^{-9}$). If the keywords for accuracy for propagation and diffusion of tracers
are equal to 10$^{-14}$, the error becomes lower than
$10^{-12}~\rm{m}^3$, thus a relative error lower than the machine
accuracy 10$^{-15}$.\\
The mass balance also shows the conservation and the amount of
discharged tracer 1, 2 and 3 is good
(e.g. for tracer 2: 10~kg$\cdot\text{m}^{-3} \times$ 2 sources $\times$ 1,100~s
$\times$ 1~m$^3\cdot\text{s}^{-1}$ = 22,000~kg with an error lower than 10$^{-5}$~kg,
so a relative error lower than 10$^{-9}$).
To get an error lower than 2.10$^{-8}$~kg, so a relative error lower
than 2.10$^{-12}$, an accuracy of 10$^{-14}$ is required for the
diffusion of tracers and propagation.
Otherwise, the mass balances may be worse but sufficient enough,
depending on the accuracy the user wishes.

\bigskip
Balance for \telkey{ACCURACY FOR PROPAGATION} $= 10^{-8}$ and
\telkey{ACCURACY FOR DIFFUSION OF TRACERS} $= 10^{-9}$ in serial:

\begin{lstlisting}[language=TelFortran]
                        FINAL MASS BALANCE
        T =        1100.0000

        --- WATER ---
        INITIAL MASS                        :     4000.000
        FINAL MASS                          :     6200.000
        MASS LEAVING THE DOMAIN (OR SOURCE) :    -2200.000
        MASS LOSS                           :    0.2391803E-06

        --- TRACER 1: TRACER 1        , UNIT : ??              * M3)
        INITIAL MASS                        :     0.000000
        FINAL MASS                          :     11000.00
        MASS EXITING (BOUNDARIES OR SOURCE) :    -11000.00
        MASS LOSS                           :    0.5897147E-05

        --- TRACER 2: TRACER 2        , UNIT : ??              * M3)
        INITIAL MASS                        :     0.000000
        FINAL MASS                          :     22000.00
        MASS EXITING (BOUNDARIES OR SOURCE) :    -22000.00
        MASS LOSS                           :    0.2584195E-05

        --- TRACER 3: TRACER 3        , UNIT : ??              * M3)
        INITIAL MASS                        :     0.000000
        FINAL MASS                          :     11000.00
        MASS EXITING (BOUNDARIES OR SOURCE) :    -11000.00
        MASS LOSS                           :   -0.3261903E-05
\end{lstlisting}

\bigskip
Balance for \telkey{ACCURACY FOR PROPAGATION} $= 10^{-8}$ and
\telkey{ACCURACY FOR DIFFUSION OF TRACERS} $= 10^{-9}$ in parallel
(4 processors):

\begin{lstlisting}[language=TelFortran]

	                FINAL MASS BALANCE
	T =        1100.0000

	--- WATER ---
	INITIAL MASS                        :     4000.000
	FINAL MASS                          :     6200.000
	MASS LEAVING THE DOMAIN (OR SOURCE) :    -2200.000
	MASS LOSS                           :    0.2391880E-06

	--- TRACER 1: TRACER 1        , UNIT : ??              * M3)
	INITIAL MASS                        :     0.000000
	FINAL MASS                          :     11000.00
	MASS EXITING (BOUNDARIES OR SOURCE) :    -11000.00
	MASS LOSS                           :    0.5897129E-05

	--- TRACER 2: TRACER 2        , UNIT : ??              * M3)
	INITIAL MASS                        :     0.000000
	FINAL MASS                          :     22000.00
	MASS EXITING (BOUNDARIES OR SOURCE) :    -22000.00
	MASS LOSS                           :    0.2584253E-05

	--- TRACER 3: TRACER 3        , UNIT : ??              * M3)
	INITIAL MASS                        :     0.000000
	FINAL MASS                          :     11000.00
	MASS EXITING (BOUNDARIES OR SOURCE) :    -11000.00
	MASS LOSS                           :   -0.3261830E-05
\end{lstlisting}

\bigskip
Balance for \telkey{ACCURACY FOR PROPAGATION} $= 10^{-14}$ and
\telkey{ACCURACY FOR DIFFUSION OF TRACERS} $= 10^{-14}$ in serial:

\begin{lstlisting}[language=TelFortran]
                        FINAL MASS BALANCE
        T =        1100.0000

        --- WATER ---
        INITIAL MASS                        :     4000.000
        FINAL MASS                          :     6200.000
        MASS LEAVING THE DOMAIN (OR SOURCE) :    -2200.000
        MASS LOSS                           :    0.4547474E-12

        --- TRACER 1: TRACER 1        , UNIT : ??              * M3)
        INITIAL MASS                        :     0.000000
        FINAL MASS                          :     11000.00
        MASS EXITING (BOUNDARIES OR SOURCE) :    -11000.00
        MASS LOSS                           :   -0.1768058E-07

        --- TRACER 2: TRACER 2        , UNIT : ??              * M3)
        INITIAL MASS                        :     0.000000
        FINAL MASS                          :     22000.00
        MASS EXITING (BOUNDARIES OR SOURCE) :    -22000.00
        MASS LOSS                           :   -0.1020817E-07

        --- TRACER 3: TRACER 3        , UNIT : ??              * M3)
        INITIAL MASS                        :     0.000000
        FINAL MASS                          :     11000.00
        MASS EXITING (BOUNDARIES OR SOURCE) :    -11000.00
        MASS LOSS                           :    0.7525159E-08
\end{lstlisting}

%\section{Conclusion}

\bigskip
To conclude, \telemac{3D} is able to compute the evolution and the conservation of
tracers discharged by sources.
