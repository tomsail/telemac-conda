% case name
\chapter{Equilibrium of a stratified flow above a bump (bump\_static)}
%
% - Purpose & Description:
%     These first two parts give reader short details about the test case,
%     the physical phenomena involved, the geometry and specify how the numerical solution will be validated
%
\section{Purpose}
%
This test case belongs to a benchmark of CFD codes with analytical solutions.
Its aim was to study the behavior of different codes, focusing on situations
where the density variations have a crucial influence on the hydrodynamical
process.
This study was carried out during Lamia Abbas's post-doctoral \cite{Abbas2015}
in 2014-2015.

This test case demonstrates the ability of \telemac{3D} to preserve
equilibria for stratified flows.
The example only shows the results for one specific 2D mesh and one distribution
of planes over the vertical.

%
\section{Description}
%
A 3~m long and 2~m wide channel with a specific bottom is considered.
The bottom elevation $z_f$is defined by a bump at the bottom:
\begin{equation}
  z_f(x,y) = \max \left(0,0.25-1.246(x-1.2)^2\right),
\end{equation}
see Figure \ref{fig:bump_static:bathy}.

\begin{figure}[H]
 \centering
 \includegraphicsmaybe{[width=0.8\textwidth]}{../img/res_z_bottom_map.png}
  \caption{Bottom elevation.}\label{fig:bump_static:bathy}
\end{figure}

The tracer used is temperature.
The density is considered as a function of the water temperature:
\begin{equation}
  \rho(T) = \rho_0 \left(1 - \alpha (T-T_0)^2\right),
\end{equation}
where $\rho_0$ = 999.972~kg.m$^{-3}$, $T_0$ = 4~$^{\circ}$C
and $\alpha$ = 7.10$^{-6}$.

There are two layers of water with different temperatures.
As the density depends on water temperature, the initial state
is chosen nearly stable \emph{i.e.} the heavier fluid is below the lighter.
Above the bump, the initial temperature condition follows the bottom elevation
as shown in Figure \ref{fig:bump_static:temp_ic:section}.

%
\section{Computational options}
%
\subsection{Mesh}
%
The triangular mesh is composed of 5,250 triangular elements (element size
$\approx$ 0.05~m) and 2,745 nodes (see Figure \ref{fig:bump_static:mesh}).

\begin{figure}[H]
 \centering
 \includegraphicsmaybe{[width=0.8\textwidth]}{../img/res_mesh.png}
  \caption{Horizontal mesh.}\label{fig:bump_static:mesh}
\end{figure}

To build the 3D mesh of prisms, 40 planes are regularly spaced over the vertical.
The vertical mesh between nodes of coordinates (0 ; 1) to (3 ; 1) can be seen
in Figure \ref{fig:bump_static:mesh:section}.

\begin{figure}[H]
 \centering
 \includegraphicsmaybe{[width=0.8\textwidth]}{../img/res_mesh_section.png}
  \caption{Vertical mesh at initial.}\label{fig:bump_static:mesh:section}
\end{figure}

The classical $\sigma$ transformation is used, but as the free surface does not
move during the computation, the mesh can be considered as fixed.\\

During Lamia Abbas's post-doctoral, 3 other 2D grids were used:
\begin{itemize}
\item one coarser grid with 1,279 elements (2D) with element size $\approx$ 0.1~m,
\item two finer grids with 22,012 resp. 132,876 elements (2D) with element
size $\approx$ 0.025~m resp. 0.01~m.
\end{itemize}

Moreover, 3 different vertical distributions were tested:
\begin{itemize}
\item two coarser vertical distributions with 10 resp. 20 horizontal planes
  leading to $\approx$ 0.1~m resp. 0.05~m vertical height,
\item one finer vertical distribution with 100 horizontal planes leading to
  $\approx$ 0.01~m.
\end{itemize}

To summarize, the different grids tested during the post-doc were:
\begin{small}
\begin{center}
\begin{tabular}{|c|c|c|c|c|c|c|c|c|}
\hline
N & 1 & 2 & 3 & 4 & 5 & 6 & 7 & 8\\
\hline 
$dl$ (m) & 0,1 & 0,05 & 0,025 & 0,01 & 0,1 & 0,1 & 0,1 & 0,05\\
\hline
$dz$ (m) & 0,1 & 0,05 & 0,025 & 0,01 & 0,05 & 0,025 & 0,01 & 0,025\\
\hline
\# elements 2D & 1,279 & 5,250 & 22,012 & 132,876 & 1,279 & 1,279 & 1,279 & 5,250\\
\hline
\# elements 3D & 11,511 & 99,750 & 858,468 & 13,154,724 & 24,301 & 49,881 & 126,621 & 204,750\\
\hline
\end{tabular}
\end{center}
\end{small}

\subsubsection{Remarks}
Meshes \#1, 2, 3 and 4 are isotropic and more and more refined.
Meshes \#1, 5, 6 and 7 (resp. 2 and 8) have the same horizontal discretisation
and a bigger and bigger number of horizontal planes (from 10 to 100).

\subsection{Physical parameters}
%
The turbulent viscosities for both velocities and tracers are constant,
also both for horizontal and vertical directions:
$\nu$ = 0.003~m$^2$/s and $\nu_T$ = 0.
%Turbulence model (both horizontal and vertical): constant model\\
%Anisotropic viscosity for velocity: 0.003~m$^2$/s\\
%Anisotropic viscosity for tracer: 0~m$^2$/s\\
No wind is taken into account.
%
\subsection{Initial and Boundary Conditions}
%
%\subsubsection{Initial conditions}
%
The initial free surface elevation is 1 m with a fluid at rest.
%Flow at rest: Constant elevation (= 1~m) and no velocity\\
The initial temperature depends on elevation, defined as:
\begin{equation}
  T_0(x,y,z) = \left\{
  \begin{tabular}{rl}
    25 $^{\circ}$C & if $z-z_f >$ 0.66~m,\\
    7  $^{\circ}$C & else
  \end{tabular}
  \right.
\end{equation}
see Figure \ref{fig:bump_static:temp_ic:section}.

\begin{figure}[H]
 \centering
 \includegraphicsmaybe{[width=0.8\textwidth]}{../img/res_temp_IC_section.png}
  \caption{Initial condition for temperature.}\label{fig:bump_static:temp_ic:section}
\end{figure}

%
%\subsubsection{Boundary conditions}
%
There are only closed lateral boundaries with free slip condition and no
friction at the bottom.
%
\subsection{General parameters}
%
The time step is 0.5~s for a simulated period of 100~s.
%
\subsection{Numerical parameters}
%
The non-hydrostatic version of \telemac{3D} is used.
To solve the advection steps, the N-type MURD scheme is chosen for the
velocities and the PSI-type MURD scheme for the temperature.
%
% - Results:
%     We comment in this part the numerical results against the reference ones,
%     giving understanding keys and making assumptions when necessary.
%
\section{Results}
%
If the diffusion of the tracer is not accounted, \emph{i.e.} if $\nu_{T} = 0$
for both horizontal and vertical directions, the system will reach a stable
equilibrium characterized by:
\begin{equation}
 \frac{\partial T}{\partial x} = 0 \quad \textrm{ and } \quad \frac{\partial T}{\partial y} = 0 \quad \forall x, y , z \in \Omega.
\end{equation}
Then the expected level of the thermocline will be $H_{eq}$ = 0.7098~m (see
\cite{Abbas2015}).

\begin{figure}[H]
 \centering
 \includegraphicsmaybe{[width=0.8\textwidth]}{../img/res_temp_section.png}
  \caption{Temperature after 100~s.}\label{fig:bump_static:temp:section}
\end{figure}

% \caption{\footnotesize Initial state and expected equilibrium.}
%
\subsubsection{Required calculations}
%\subsubsection{Remarks}
A bias exists in the reference solution because the initial quantities of warm
and cold waters depend on the 3D mesh. This bias is constant and decreases when
refining the mesh.

Moreover, it is difficult to determine the exact location of the thermocline
with computations.
Indeed, due to numerical diffusion and interpolations done by the code or
post-treatment tools, the mixing layer is more or less big.
For all these reasons, two points can be investigated:
\begin{itemize}
\item A comparison between the distribution of the temperature at the equilibrium
  and the expected position of the thermocline $H_{eq}$,
% \caption{\footnotesize TELEMAC-3D results after \SI{100}{s}. The black line is the reference solution.}
\item Evaluation of the diffusion by mesuring the mixed water defined as:
  $Q_m = \int_{0}^{L} (Z_2 - Z_1) dx$
  with $Z_1$ and $Z_2$ defined such as $T(x,Z_1)$ = 10~$^{\circ}$C
  and $T(x,Z_2)$ = 22~$^{\circ}$C.
% \caption{\footnotesize Comparison of the mixed layer.}
\end{itemize}

Here, as we do not compare different codes or meshes, we only look at the
location of the expected thermocline.

\subsubsection{Planes distribution}
Results for the different $\sigma$ planes distribution are not discussed here,
but the user can refer to \cite{Abbas2015} for more informations.

A suitable distribution of horizontal planes (a $\sigma$ transformation for the
first 3 bottom planes and $z$ for the ones above so that a maximum number of
horizontal planes around the thermocline) enables to greatly improve the results
with a reasonable number of elements.

The Automatic Mesh Refinement option (\telkey{MESH TRANSFORMATION} = 5) enables
to save the use of some extra planes and may give correct results for such
applications.

See \cite{Abbas2015} for more informations.
%
\section{Conclusion}
%
\telemac{3D} is able to preserve equilibria for stratified flows.
