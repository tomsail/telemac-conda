\chapter{Stability of a stratified flow (stratification)}

\section{Description}
\bigskip
This test is a flat channel with saline stratification.
It demonstrates the ability of \telemac{3D} to model stratified
flow with a special focus on the stability of the stratification.
This case also demonstrates the capacity of the $k$-$\epsilon$ model
to reproduce turbulent phenomena.

\bigskip
The domain is a rectangular channel with 2,000~m long and 100~m wide 
(see Figure \ref{t3d:stratification:fig:bathy}). The bottom of 
this channel has a mild slope (0~m at the entrance, -0.019~m at the output).
The general water depth is 10~m and the constant velocity 
along the channel is imposed. Salinity (or tracer) is prescribed 
as shown in Figure \ref{t3d:stratification:fig:salInit}.
Density law depends then on salinity ($S_{al}$):
\begin{equation*}
\rho = \rho_{ref} (1 + 750.10^{-6} . S_{al} ),
\rm{with }~\rho_{ref} = 999.972~\rm{kg.m}^{-3}
\end{equation*}
\\
Note that the turbulent viscosity is constant in horizontal direction
and equal to $0.1~\text{m}^2.\text{s}^{-1}$. The $k$-$\epsilon$~model 
is only used in the vertical direction. 

\begin{figure}[!htbp]
 \centering
 \includegraphicsmaybe{[width=\textwidth]}{../img/bathy.png}
 \caption{Bottom topography.}
 \label{t3d:stratification:fig:bathy}
\end{figure}
\begin{figure}[!htbp]
 \centering
 \includegraphicsmaybe{[width=\textwidth]}{../img/Sal_ini.png}
 \caption{Initial salinity along the channel.}
 \label{t3d:stratification:fig:salInit}
\end{figure}

\subsection{Initial and Boundary Conditions}
\bigskip
The initial water depth is 10~m with a constant longitudinal 
velocity equal to 1~m.s$^{-1}$.\\
The tracer (or salinity) is equal to 38~kg.m$^{-3}$ (g.L$^{-1}$)
at the bottom below the plane number 18 
(i.e. at 2.5~m of the water surface) and 28~kg.m$^{-3}$ 
at the top above the plane number 18 
(Figure \ref{t3d:stratification:fig:salInit}).
Note that the tracer and velocity field are directly initialised in the
\telfile{USER\_CONDI3D\_UVW} and \telfile{USER\_CONDI3D\_TRAC} subroutines.

\bigskip
The boundary conditions are:
\begin{itemize}
\item For the solid lateral walls, a slip condition 
on the channel banks is used for the velocity,
\item On the bottom, Strickler law with friction coefficient equal to 
70~$\text{m}^{1/3}.\text{s}^{-1}$ is imposed,
\item Upstream, a flow rate equal to 1,000~$\text{m}^{3}.\text{s}^{-1}$ is imposed
and a tracer equal to the initial repartition,
\item Downstream, the water depth is imposed at 9.981054~m.
\end{itemize}
Note that the tracer and velocity field are directly prescribed on the liquid
boundaries in the new (since V7P3) \telfile{USER\_BORD3D} subroutine.

\subsection{Mesh and numerical parameters}
\bigskip
The mesh (Figure \ref{t3d:stratification:fig:meshH} and \ref{t3d:stratification:fig:meshV})  
is composed of 2,204 triangular elements (1,188 nodes) with 21 planes  
regularly spaced on the vertical, to form prism elements.\\

\begin{figure}[!htbp]
 \centering
 \includegraphicsmaybe{[width=\textwidth]}{../img/Mesh.png}
 \caption{Horizontal mesh.}
 \label{t3d:stratification:fig:meshH}
\end{figure}
\begin{figure}[!htbp]
 \centering
 \includegraphicsmaybe{[width=\textwidth]}{../img/meshV.png}
 \caption{Vertical mesh.}
 \label{t3d:stratification:fig:meshV}
\end{figure}

\bigskip
The time step is 2~s for a simulated period of 2,000~s. 

\bigskip
This case is computed with the hydrostatic pressure hypothesis.  To solve advection, the N-type MURD scheme (scheme 4) 
is used for the velocities, $k$-$\epsilon$ model and the tracer (or salinity). 
The implicitation coefficients for depth and velocities are equal to 1.\\

\section{Results}

\bigskip
As expected, the vertical gradient of salinity remains well stable, see Figures
\ref{t3d:stratification:fig:salInit} and \ref{t3d:stratification:fig:sal2000}.

\bigskip
In Figure \ref{t3d:stratification:fig:evolTurb}, we can observe 
the turbulence phenomenon modelled by the $k$-$\epsilon$ model.
This turbulence is created at the bottom and is developing on the
vertical column of water.
\begin{figure}[!htbp]
 \centering
 \includegraphicsmaybe{[width=1\textwidth]}{../img/Sal_f.png}
 \caption{Final salinity along the channel at 2,000~s.}
 \label{t3d:stratification:fig:sal2000}
\end{figure}

However, the turbulence is clearly blocked by the saline stratification 
(see Figure \ref{t3d:stratification:fig:visco}).

\begin{figure}[!htbp]
 \centering
 \includegraphicsmaybe{[width=1\textwidth]}{../img/turb.png}
 \caption{Evolution of turbulence on the vertical along the channel.}
 \label{t3d:stratification:fig:evolTurb}
\end{figure}

%\section{Conclusion}

\bigskip
To conclude, \telemac{3D} is able to represent correctly stratified flows.
In addition, the $k$-$\epsilon$ model is able to simulate turbulence
generated by bottom friction.

\begin{figure}[!htbp]
 \centering
 \includegraphicsmaybe{[width=0.85\textwidth]}{./img/visco_profile.png}
 \caption{Viscosity profile along the vertical at 1,250~m.}
 \label{t3d:stratification:fig:visco}
\end{figure}
