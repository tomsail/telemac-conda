\chapter{Weirs}

\section{Purpose}
These two test cases present a flow over different sills
treated as singularities. They allow to show
that \telemac{2D} is able to treat a number of flow singularities
as internal boundary conditions. Moreover, they allow also to check
the tracers function for this type of problem in \telemac{2D}.

\section{Description}

The sills may be considered under the traditional approach made
in open channel hydraulics through a relation between two superimposed
open boundaries, short sills may be treated as a couple of nodes
with respective source and sink terms.\\

This sills description, which are represented in a channel as internal
singularities, was introduced in \telemac{2D} in order to avoid the
multiplication of computational nodes and associated reduction in time
step when a sill is represented thanks to variations in the bathymetry.\\
In the first version implemented in \telemac{2D}, the weir law as traditionally
used in channel hydraulics is prescribed through two boundary conditions:
one upstream the weir and one downstream (with the same number of nodes
upstream and downstream).
Recently, a more generic solution has been implemented where the weir law
is prescribed through sources points and where the description of the weir
itself could be done at a smaller scale than the surrounding mesh.
It should be mentioned however that both options give satisfactory results
only if the flow is relatively perpendicular to the weir, which is
the case in the two tests.

\section{Option 1 - Horizontal weirs}

\subsection{Geometry and mesh}

In this test case, the sills  are so represented as internal
singularities, as just indicated.
The geometry dimensions of rectangular channel are 848~m wide
and  3,522~m long. The channel is flat bottom and it is decomposed
of four part reaches 848~m long. The three upstream reaches are limited
at their downstream end by 3 sills with crest heights 1.8~m (upstream sill),
1.6~m and 1.4~m (downstream sill).

The mesh is regular along the domain. It is generally made up
with quadrangles split into two triangles. It is irregular in
the upstream reach in order to test the sensitivity on this
feature of the flow results above the sill (see Figure \ref{t2d:weirs:fig:geo}).
It is composed of 870 triangular elements (519 nodes) and
the size of triangles ranges between 53~m and 120~m.

\begin{figure}[!htbp]
 \centering
 \includegraphicsmaybe{[width=0.6\textwidth]}{../img/geo.png}
 \caption{Mesh and topography of the domain.}
 \label{t2d:weirs:fig:geo}
\end{figure}

\subsection{Initial conditions}

The initial conditions are a null velocity,
a water depth of 1.35~m and a tracer value of 50.

\subsection{Boundary conditions}

The boundary conditions (Figure \ref{t2d:weirs:fig:geo}) are:
\begin{itemize}
\item At the channel entrance (lower left), the flow rate is
$Q$ = 600~m$^3$s$^{-1}$ and the tracer value is 100.
\item At the channel outlet (upper left), the water depth is
$h$ = 1.35~m.
\end{itemize}

\subsection{Physical parameters}

On  bottom friction, the Strickler formula
with friction coefficient equal to
30~m$^{1/3}$.s$^{-1}$ is imposed.
No friction is taken into account on lateral walls.
Note that the turbulent viscosity is constant and equal to 1~m$^2$.s$^{-1}$.

\subsection{Numerical parameters}

The time step is 150~s for a period of 6,000~s.
The resolution accuracy for the velocity is taken at $10^{-10}$.
Note that for numerical resolution, GMRES (Generalized Minimal Residual Method)
is used for solving the propagation step (option 7). To solve advection,
the characteristics scheme (scheme 1), and the conservative scheme (scheme 5)
is used respectively for the velocities and for the depth. To finish,
the implicitation coefficients for depth and velocities are equal to 0.55
(default values).\\
The triangular elements types are linear triangles (P1, 3 values per element,
the corners) for water depth and quasi-bubble triangle (4 values per element,
the corners and the element center) for velocities.
It should also be noted that a tracer is used.
For the advection resolution, it is used a conservative N-scheme (scheme \#4)
and the solver for diffusion of tracer is the conjugate gradient (option 1).

\subsection{Results}

As it can be seen in Figure \ref{t2d:weirs:velocity},
the velocity field remains regular laterally in the different reaches of the channel.

\begin{figure}[!htbp]
\begin{minipage}[t]{0.50\textwidth}
 \centering
 \includegraphicsmaybe{[width=\textwidth]}{../img/figure_velo1500s.png}
\end{minipage}
\begin{minipage}[t]{0.50\textwidth}
 \centering
 \includegraphicsmaybe{[width=\textwidth]}{../img/figure_velo3000s.png}
\end{minipage}
\begin{minipage}[t]{0.50\textwidth}
 \centering
 \includegraphicsmaybe{[width=\textwidth]}{../img/figure_velo4500s.png}
\end{minipage}
\begin{minipage}[t]{0.50\textwidth}
 \centering
 \includegraphicsmaybe{[width=\textwidth]}{../img/figure_velo6000s.png}
\end{minipage}
 \caption{Evolution of velocity field in time.}
 \label{t2d:weirs:velocity}
\end{figure}

The water level increases progressively as expected
in the three upstream reaches during the simulated period
(see Figure \ref{t2d:weirs:fig:z}). The relations between
the discharge (per unit of width) on the sill, the water levels
upstream and downstream and the sill crest elevation must be respected.
They are:
\begin{itemize}
\item Free overflow weir:
\begin{equation*}
q=\mu \sqrt{2g}\left( z_{up}-z_{sill} \right)^{3/2},
\end{equation*}
\item Drowned weir :
 	\begin{equation*}
q=\frac{2}{3\sqrt{3}}C_{d} \sqrt{2g}\left( z_{down}-z_{sill} \right)\sqrt{\left( z_{up}-z_{down} \right)}.
\end{equation*}
\end{itemize}
The transition from free overflow to drowned condition is defined by:
\begin{equation}
  	z_{down} \leq z_{sill}+ \frac{2}{3}\left( z_{up}-z_{sill} \right),
  	\label{t2d:weirs:eq:1}
\end{equation}
Where $z_{up}$, $z_{down}$ and $z_{sill}$ are respectively
water level upstream (m), water level downstream (m) and
sill crest elevation (m).  $q$ is discharge per unit width
(m$^2$.s$^{-1}$). $g$ is the gravitational acceleration (m.s$^{-2}$).
$\mu$ is viscosity coefficient (Pa.s$^{-1}$) and $C_d$ is the discharge coefficient
(usually between 0.4 and 0.5).

\begin{figure}[H]
 \centering
 \includegraphicsmaybe{[width=0.85\textwidth]}{../img/free_surface.png}
 \caption{Evolution of the free surface elevation in time.}
 \label{t2d:weirs:fig:z}
\end{figure}

The \telemac{2D} results respect well these relations \eqref{t2d:weirs:eq:1}
(Figure  \ref{t2d:weirs:fig:z}). Furthermore the tracer propagation is well
 carried out through internal singularities (Figure \ref{t2d:weirs:fig:tracer}).

\begin{figure}[H]
 \centering
 \includegraphicsmaybe{[width=0.85\textwidth]}{../img/tracer.png}
 \caption{Evolution of the tracer in time.}
 \label{t2d:weirs:fig:tracer}
\end{figure}

\section{Option 2 - Generic weirs}

\subsection{Geometry and mesh}

In this second test case, the sills are also represented as internal
singularities, but with different shape, along profile and number of points.

The geometry is composed by 6 different squares 1,000~m $\times$ 1,000~m.
The mesh of each square is irregular with mean size equal to 50~m (domain 4), 
100~m (domain 1, 3 and 5) and 200~m (domain 2 and 6) (see Figure \ref{t2d:weirs2:fig:mesh}).

There is 5 weirs between those 6 small domains
(see Figure \ref{t2d:weirs2:fig:geo}).

The global mesh is composed of 1,702 triangular elements (977 nodes).

\begin{figure}[H]
 \centering
 \includegraphicsmaybe{[width=0.6\textwidth]}{../img/weirs2_geo.png}
 \caption{Mesh of the domain.}
 \label{t2d:weirs2:fig:mesh}
\end{figure}

\begin{figure}[H]
 \centering
 \includegraphicsmaybe{[width=0.6\textwidth]}{img/geometry.png}
 \caption{Topography of the domain with weirs position.}
 \label{t2d:weirs2:fig:geo}
\end{figure}

Those 5 weirs have different shapes (see Figure \ref{t2d:weirs2:fig:weirsprofile}):
\begin{itemize}
\item Simple horizontal shape between domain 1 and 2, %(see Figure \ref{t2d:weirs2:fig:w1}).
\item Horizontal shape but with a zigzag trace between domain 2 and 3 
that leads to a longer weir, %(see Figure \ref{t2d:weirs2:fig:w2}).
\item V shape between domain 3 and 4, %(see Figure \ref{t2d:weirs2:fig:w3}).
\item Horizontal shape with 2 slots between domain 4 and 5, %(see Figure \ref{t2d:weirs2:fig:w4}).
\item Free Shape between domain 5 and 6. %(see Figure \ref{t2d:weirs2:fig:w5}).
\end{itemize}

\begin{figure}[!htbp]
\begin{minipage}[t]{0.50\textwidth}
 \centering
 \includegraphicsmaybe{[width=\textwidth]}{img/weir_1-2.png}
 \captionof{subfigure}{Profile of weirs between 1 and 2}
\end{minipage}
\begin{minipage}[t]{0.50\textwidth}
 \centering
 \includegraphicsmaybe{[width=\textwidth]}{img/weir_2-3.png}
 \captionof{subfigure}{Profile of weirs between 2 and 3}
\end{minipage}
\begin{minipage}[t]{0.50\textwidth}
 \centering
 \includegraphicsmaybe{[width=\textwidth]}{img/weir_3-4.png}
 \captionof{subfigure}{Profile of weirs between 3 and 4}
\end{minipage}
\begin{minipage}[t]{0.50\textwidth}
 \centering
 \includegraphicsmaybe{[width=\textwidth]}{img/weir_4-5.png}
 \captionof{subfigure}{Profile of weirs between 4 and 5}
\end{minipage}
\begin{minipage}[t]{\textwidth}
 \centering
 \includegraphicsmaybe{[width=0.50\textwidth]}{img/weir_5-6.png}
 \captionof{subfigure}{Profile of weirs between 5 and 6}
\end{minipage}
 \caption{Profile of weirs.}
 \label{t2d:weirs2:fig:weirsprofile}
\end{figure}

\newpage 
\subsection{Initial conditions}

The initial conditions are a null velocity, null concentration tracer,
a water depth of 1~m in domain 1 and 6 and 0.1~m in other domain.

\subsection{Boundary conditions}

The boundary conditions are:
\begin{itemize}
\item At the domain 1 entrance (left side), the flow rate starts from 
$Q$ = 0~m$^3$s$^{-1}$ and increases to $Q$ = 500~m$^3$s$^{-1}$
during the 200 first seconds and remain to this value after that.
The tracer value is 100.
\item At the channel outlet, the water depth is $h$ = 1m.
\end{itemize}

\subsection{Physical parameters}

On  bottom friction, the Strickler formula
with friction coefficient equal to
50~m$^{1/3}$.s$^{-1}$ is imposed.
No friction is taken into account on lateral walls.
For turbulence, the default parameters are used.

\subsection{Numerical parameters}

The time step is 10~s for a period of 43,200~s (= 12~h).
All other numerical parameters are let at theirs default value.

\subsection{Results}

As it can be seen in Figure \ref{t2d:weirs2:velocity},
the velocity field remains regular laterally between the 3 first domains.
Between domain 3 and 4, the flow shows a concentration in the center due to the
V shape and between 4 and 5, the flow is splitted in 2 according to the slots.
Over the last weir, the flow is irregular but smoothed by the large cell sized
of the domain...

\begin{figure}[!htbp]
\begin{minipage}[t]{0.50\textwidth}
 \centering
 \includegraphicsmaybe{[width=\textwidth]}{../img/weirs2_figure_velo_00600s.png}
\end{minipage}
\begin{minipage}[t]{0.50\textwidth}
 \centering
 \includegraphicsmaybe{[width=\textwidth]}{../img/weirs2_figure_velo_03600s.png}
\end{minipage}
\begin{minipage}[t]{0.50\textwidth}
 \centering
 \includegraphicsmaybe{[width=\textwidth]}{../img/weirs2_figure_velo_07200s.png}
\end{minipage}
\begin{minipage}[t]{0.50\textwidth}
 \centering
 \includegraphicsmaybe{[width=\textwidth]}{../img/weirs2_figure_velo_10800s.png}
\end{minipage}
\begin{minipage}[t]{0.50\textwidth}
 \centering
 \includegraphicsmaybe{[width=\textwidth]}{../img/weirs2_figure_velo_14400s.png}
\end{minipage}
\begin{minipage}[t]{0.50\textwidth}
 \centering
 \includegraphicsmaybe{[width=\textwidth]}{../img/weirs2_figure_velo_18000s.png}
\end{minipage}
\begin{minipage}[t]{0.50\textwidth}
 \centering
 \includegraphicsmaybe{[width=\textwidth]}{../img/weirs2_figure_velo_43200s.png}
\end{minipage}
 \caption{Evolution of velocity field in time.}
 \label{t2d:weirs2:velocity}
\end{figure}

The final discharge for each weirs is presented in the following table. The test case
is supposed to reach steady state at the end of simulation so we should obtain the same discharge
as imposed upstream ($Q$ = 500~m$^3$s$^{-1}$).

\begin{table}[H]
\centering
\begin{tabular}{|c|c|}
 \hline Weir number & Discharge \\
 \hline
 \InputIfFileExists{../img/qweirs.txt}{}{}\\
 \hline
\end{tabular}
\label{t2d:weirs2:tab1}
\caption{Weirs2 test case: final discharge over weirs.}
\end{table}

The evolution of piece-wise discharge (the discharge on each elemental part of
the weir) can be seen in Figure \ref{t2d:weirs2:fig:weirsdischarge}.

\begin{itemize}
\item For weirs 1 and 2, which are horizontal, that the piece-wise discharge is
  the same for each element of the weir,
\item For weirs 3, the highest discharge is located in the middle part of the
  weir and the discharge is 0 on the lateral part,
\item For weir 4, only the element in the slot have a positive discharge,
\item For weir 5, the discharge could be linked to the shape of the weir.
\end{itemize}

\begin{figure}[!htbp]
\begin{minipage}[t]{0.50\textwidth}
 \centering
 \includegraphicsmaybe{[width=\textwidth]}{../img/q_weir1.png}
 \captionof{subfigure}{Elemental discharge of weir 1}
\end{minipage}
\begin{minipage}[t]{0.50\textwidth}
 \centering
 \includegraphicsmaybe{[width=\textwidth]}{../img/q_weir2.png}
 \captionof{subfigure}{Elemental discharge of weir 2}
\end{minipage}
\begin{minipage}[t]{0.50\textwidth}
 \centering
 \includegraphicsmaybe{[width=\textwidth]}{../img/q_weir3.png}
 \captionof{subfigure}{Elemental discharge of weir 3}
\end{minipage}
\begin{minipage}[t]{0.50\textwidth}
 \centering
 \includegraphicsmaybe{[width=\textwidth]}{../img/q_weir4.png}
 \captionof{subfigure}{Elemental discharge of weir 4}
\end{minipage}
\begin{minipage}[t]{\textwidth}
 \centering
 \includegraphicsmaybe{[width=0.50\textwidth]}{../img/q_weir5.png}
 \captionof{subfigure}{Elemental discharge of weir 5}
\end{minipage}
 \caption{Evolution of discharge per elemental part of each weir.}
 \label{t2d:weirs2:fig:weirsdischarge}
\end{figure}

The water level increases progressively as expected
in the different domains during the simulated period
(see Figure \ref{t2d:weirs2:fig:z}).
Like in Option 1, the relations \eqref{t2d:weirs:eq:1} are respected.

\begin{figure}[H]
 \centering
 \includegraphicsmaybe{[width=0.85\textwidth]}{../img/weirs2_free_surface.png}
 \caption{Evolution of the free surface elevation in time.}
 \label{t2d:weirs2:fig:z}
\end{figure}

The tracer propagation is also well carried out through internal singularities
 (Figure \ref{t2d:weirs2:fig:tracer}).

\begin{figure}[H]
 \centering
 \includegraphicsmaybe{[width=0.85\textwidth]}{../img/weirs2_tracer.png}
 \caption{Evolution of the tracer in time.}
 \label{t2d:weirs2:fig:tracer}
\end{figure}

\section{Conclusion}
To conclude, \telemac{2D} computes adequately weir
flows as given by analytical hydraulic laws.
This type of flow is represented as an internal
 singularity in the model.
These internal singularity could be described with 2 different options:
\begin{itemize}
\item Option 1, the weir law as traditionally used in channel hydraulics
 is prescribed through boundary conditions. This option requires the same
 number of nodes upstream and downstream and the weir should be horizontal,
\item Option 2, the weir law is prescribed through sources points located
 on the upstream and downstream boundary. The description of the weir itself
 could be done at a smaller scale than the surrounding mesh without any
 constraint on its level.
\end{itemize}
