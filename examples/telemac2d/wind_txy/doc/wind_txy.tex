\chapter{Wind depending on time and space (wind\_txy)}
%
\section{Purpose}

This test case presents the hydrodynamics study resulting from wind set-up in a
closed rectangular basin or channel.
Contrary to the wind example, here the wind depends on time and space
(\telkey{OPTION FOR WIND} = 3).
2 examples are given:
\begin{itemize}
\item the first one uses an atmospheric data file written in ASCII,
\item the second one uses an atmospheric data file written in binary.
\end{itemize}
These are examples on how to manage wind or meteo data if depending on time
and space and not standardly done by \telemac{2D}.

\section{Description}

\subsection{Geometry and mesh}

This test case models the hydrodynamics behaviours due to wind blowing 
in a closed rectangular basin. The geometry dimensions of basin are 100~m wide 
and  500~m long. The basin has a flat bottom and the water depth is equal to 2~m depth. 
The mesh is irregular in the basin. The mesh is shown in Figure \ref{t2d:windtxy:fig:mesh}. 
It is composed of 551 triangular elements (319 nodes) and the size of triangles
ranges between 14~m and 24~m.

\begin{figure}[!htbp]
 \centering
 \includegraphicsmaybe{[width=0.9\textwidth]}{../img/Mesh.png}
 \caption{Mesh.}
 \label{t2d:windtxy:fig:mesh}
\end{figure}

\subsection{Initial conditions}

The computation is a continuation of the wind example with steady state with constant wind
blowing on the channel.

\subsection{Boundary conditions}

The boundary conditions are:
\begin{itemize}
\item For the solid walls, a slip condition in the basin is used for the velocity,
\item No bottom friction,
\item The wind stress at the surface is imposed.
\end{itemize}
The wind conditions allowing to compute the wind stress are :
\begin{itemize}
\item The wind velocity is interpolated depending on 5 stations at coordinates (0;0), (500;0), (500;100), (0;100), (250;50) and linearly depending on time. The interpolation is implemented in the user Fortran subroutines,
\item The coefficient of wind influence is
 $a_{wind} \frac{\rho_{air}}{\rho_{water}} = 1.2615 \cdot 10^{-3}$ 
 with $\rho_{air}$, $\rho_{water}$ which are respectively the air density 
 and the water density and with $a_{wind}$ an adimentional coefficient.
\end{itemize}

\subsection{Physical parameters}

The turbulent viscosity is constant with velocity 
diffusivity equal to 0~m$^2$/s.

As the wind depends on time and space and there is no unique way to manage it,
the treatment of wind conditions is done by changing the standard \telfile{METEO}
subroutine and adding a new subroutine called \telfile{IDWM\_T2D}.
Moreover, as the formats of the \telkey{BINARY ATMOSPHERIC DATA FILE} or the
\telkey{ASCII ATMOSPHERIC DATA FILE} are not the one managed by the
\telfile{METEO\_TELEMAC} module, the keyword
\telkey{FREE FORMAT FOR ATMOSPHERIC DATA FILE} is activated.
In the case of the \telkey{BINARY ATMOSPHERIC DATA FILE}, the two keywords
\telkey{ORIGINAL DATE OF TIME} and \telkey{ORIGINAL HOUR OF TIME} have to be
filled in.

\subsection{Numerical parameters}

The time step is 5~s. The simulation duration is then 100~s after the restart.
For numerical resolution, conjugate gradient on a normal equation 
is used for solving the propagation step (option 3 = default).
The characteristics scheme (scheme 1) is used to solve the advection of velocities.
The implicitation coefficients for depth and velocities are equal to 0.5.

\section{Results}

Figures \ref{t2d:windtxy:fig:FreeSurf0}, \ref{t2d:windtxy:fig:FreeSurf50s},
\ref{t2d:windtxy:fig:FreeSurftf} show the free surface field at initial time,
after 50~s and at final time (= 100~s) for the ASCII format.
Figures \ref{t2d:windtxy:fig:Velo0}, \ref{t2d:windtxy:fig:Velo50s},
\ref{t2d:windtxy:fig:Velotf} show the velocity field and streamlines
at initial time, after 50~s and at final time (= 100~s) for the ASCII format.
Figures \ref{t2d:windtxy:fig:FreeSurfbin0}, \ref{t2d:windtxy:fig:FreeSurfbin50s},
\ref{t2d:windtxy:fig:FreeSurfbintf} show the free surface field at initial time,
after 50~s and at final time (= 100~s) for the binary format.
Figures \ref{t2d:windtxy:fig:Velobin0}, \ref{t2d:windtxy:fig:Velobin50s},
\ref{t2d:windtxy:fig:Velobintf} show the velocity field and streamlines at
initial time, after 50~s and at final time (= 100~s) for the binary format.

\begin{figure}[H]
 \centering
 \includegraphicsmaybe{[width=0.9\textwidth]}{../img/FreeSurface0.png}
 \caption{Free surface along basin at initial time step (ASCII format).}
 \label{t2d:windtxy:fig:FreeSurf0}
\end{figure}

\begin{figure}[H]
 \centering
 \includegraphicsmaybe{[width=0.9\textwidth]}{../img/FreeSurface50s.png}
 \caption{Free surface along basin at 50~s (ASCII format).}
 \label{t2d:windtxy:fig:FreeSurf50s}
\end{figure}

\begin{figure}[H]
 \centering
 \includegraphicsmaybe{[width=0.9\textwidth]}{../img/FreeSurfacetf.png}
 \caption{Free surface along basin at final time step (ASCII format).}
 \label{t2d:windtxy:fig:FreeSurftf}
\end{figure}

\begin{figure}[H]
 \centering
 \includegraphicsmaybe{[width=0.9\textwidth]}{../img/VelocityStream0.png}
 \caption{Wind velocity and streamlines at initial time step (ASCII format).}
 \label{t2d:windtxy:fig:Velo0}
\end{figure}

\begin{figure}[H]
 \centering
 \includegraphicsmaybe{[width=0.9\textwidth]}{../img/VelocityStream50s.png}
 \caption{Wind velocity and streamlines at 50~s (ASCII format).}
 \label{t2d:windtxy:fig:Velo50s}
\end{figure}

\begin{figure}[H]
 \centering
 \includegraphicsmaybe{[width=0.9\textwidth]}{../img/VelocityStreamtf.png}
 \caption{Wind velocity and streamlines at final time step (ASCII format).}
 \label{t2d:windtxy:fig:Velotf}
\end{figure}

\begin{figure}[H]
 \centering
 \includegraphicsmaybe{[width=0.9\textwidth]}{../img/FreeSurfacebin0.png}
 \caption{Free surface along basin at initial time step (binary format).}
 \label{t2d:windtxy:fig:FreeSurfbin0}
\end{figure}

\begin{figure}[H]
 \centering
 \includegraphicsmaybe{[width=0.9\textwidth]}{../img/FreeSurfacebin50s.png}
 \caption{Free surface along basin at 50~s (binary format).}
 \label{t2d:windtxy:fig:FreeSurfbin50s}
\end{figure}

\begin{figure}[H]
 \centering
 \includegraphicsmaybe{[width=0.9\textwidth]}{../img/FreeSurfacebintf.png}
 \caption{Free surface along basin at final time step (binary format).}
 \label{t2d:windtxy:fig:FreeSurfbintf}
\end{figure}

\begin{figure}[H]
 \centering
 \includegraphicsmaybe{[width=0.9\textwidth]}{../img/VelocityStreambin0.png}
 \caption{Wind velocity and streamlines at initial time step (binary format).}
 \label{t2d:windtxy:fig:Velobin0}
\end{figure}

\begin{figure}[H]
 \centering
 \includegraphicsmaybe{[width=0.9\textwidth]}{../img/VelocityStreambin50s.png}
 \caption{Wind velocity and streamlines at 50~s (binary format).}
 \label{t2d:windtxy:fig:Velobin50s}
\end{figure}

\begin{figure}[H]
 \centering
 \includegraphicsmaybe{[width=0.9\textwidth]}{../img/VelocityStreambintf.png}
 \caption{Wind velocity and streamlines at final time step (binary format).}
 \label{t2d:windtxy:fig:Velobintf}
\end{figure}

\section{Conclusion}
\telemac{2D} is able to compute wind generated flows when depending on time and space.
