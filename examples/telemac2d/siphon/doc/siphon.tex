\chapter{Culvert modelling (siphon)}
\section{Purpose}

To demonstrate that \telemac{2D} can solve the flow in a culvert considered as
an internal singularity, under the form of a couple of source and sink nodes.
Also to show that \telemac{2D} can compute tracer dispersion.

%\section{Approach}
\section{Description}

Two square tanks are connected hydraulically by a culvert. The culvert is
represented by a couple of source / sink nodes, one in each tank. The water
level and the tracer concentration are initially higher in the left tank.

\subsection{Geometry and mesh}

\begin{itemize}
\item  Two identical square tanks are located 100~m apart,
\item  The dimension of each square tank is 200~m $\times$ 200~m,
\item  The water depth at rest is 4~m in the left tank, and 2~m in the right tank.
\end{itemize}

The mesh is regular. It is made up with squares split into 2 triangles.

\begin{itemize}
\item 1,600 triangular elements,
\item 882 nodes,
\item Maximum size range: $\sqrt{200}$ = 14.14~m.
\end{itemize}

\subsection{Boundaries}

\begin{itemize}
\item Solid walls with slip condition in the domain.
\end{itemize}

Bottom:

\begin{itemize}
\item Strickler formula with friction coefficient = 20~m$^{1/3}$/s.
\end{itemize}

\subsection{Physical Parameters}

Turbulence: Model of constant viscosity with velocity diffusivity = 1~m$^2$/s.

\subsection{Numerical Parameters}

Type of advection:
\begin{itemize}
\item Characteristics on velocities (scheme \#1),
\item Conservative + modified SUPG on depth (mandatory),
\item Centred semi-implicit scheme + SUPG decentring on tracer (scheme \#2).
\end{itemize}

Type of element:
\begin{itemize}
\item Quasi-bubble triangle for velocities,
\item Linear triangle P1 for $h$.
\end{itemize}

\begin{itemize}
\item GMRES solver,
\item Accuracy = 10$^{-8}$.
\end{itemize}

Tracer:

\begin{itemize}
\item Initial concentrations: 100~\% in the left square, and 50~\% in the right square,
\item Co-ordinates of source / sink nodes: left = (100;100) and right = (400;100),
\item No water discharge of sources,
\item GMRES solver.
\end{itemize}

Time data:

\begin{itemize}
\item Time step = 2.5~s,
\item Simulation duration = 600~s.
\end{itemize}

Mesh and initial state are shown in Figure \ref{fig:siphon:mesh}.

\section{Results}

The water flows from the left tank to the right tank through the culvert. The
water level decreases regularly in the left tank and increases regularly in the
right tank (Figure \ref{fig:siphon:evol}). Simultaneously, a spot of tracer with concentration
100 arrives in the right tank and disperses.

In the left tank, the culvert is vertical. Therefore, the flow is regular and
symmetric around the sink node. In the right tank, the culvert is horizontal in
the direction of $y$-axis. Therefore, the flow takes this direction from the
source node and the velocity field forms two eddies around the source (Figure
\ref{fig:siphon:evol_vel}).

Water mass and tracer mass are conserved: no water mass is lost whereas the
cumulated loss of tracer mass at the end of the simulation is below 0.9~\% of
the initial mass.

\section{Conclusions}

TELEMAC-2D can be used for the treatment of an internal singularity, such as a
culvert, and also for the treatment of dispersion by currents and diffusion of
a tracer.

\section{Figures}

\begin{figure}
\centering
\includegraphicsmaybe{[width=0.85\textwidth]}{../img/Mesh.png}
\caption{Mesh of the domain.}
\label{fig:siphon:mesh}
\end{figure}
\begin{figure}
\centering
\includegraphicsmaybe{[width=0.85\textwidth]}{../img/tracor1_init.png}
\caption{Initial state.}\label{fig:siphon:init}
\end{figure}


\begin{figure}
\centering
\includegraphicsmaybe{[width=0.85\textwidth]}{../img/profile_tracer.png}
\includegraphicsmaybe{[width=0.85\textwidth]}{../img/free_surface.png}
 \caption{Evolution of the tracer concentration, and evolution of the free surface elevation in time in both tanks.}\label{fig:siphon:evol}
\end{figure}

\begin{figure}
\centering
\includegraphicsmaybe{[width=0.85\textwidth]}{../img/tracer_100.png}
\includegraphicsmaybe{[width=0.85\textwidth]}{../img/tracer_200.png}
\includegraphicsmaybe{[width=0.85\textwidth]}{../img/tracer_300.png}
 \caption{Evolution of the tracer concentration in time in both tanks.}\label{fig:siphon:evolbis}
\end{figure}

\begin{figure}
\centering
\includegraphicsmaybe{[width=0.85\textwidth]}{../img/tracer_400.png}
\includegraphicsmaybe{[width=0.85\textwidth]}{../img/tracer_500.png}
\includegraphicsmaybe{[width=0.85\textwidth]}{../img/tracer_600.png}
 \caption{Evolution of the tracer concentration in time in both tanks.}\label{fig:siphon:evolbis2}
\end{figure}

%\begin{figure}
%\centering
%\includegraphicsmaybe{[width=0.85\textwidth]}{../img/free_surf_100.png}
%\includegraphicsmaybe{[width=0.85\textwidth]}{../img/free_surf_200.png}
%\includegraphicsmaybe{[width=0.85\textwidth]}{../img/free_surf_300.png}
%\includegraphicsmaybe{[width=0.85\textwidth]}{../img/free_surf_400.png}
%\includegraphicsmaybe{[width=0.85\textwidth]}{../img/free_surf_500.png}
%\includegraphicsmaybe{[width=0.85\textwidth]}{../img/free_surf_600.png}
% \caption{Evolution of free surface in time in both tanks.}\label{fig:siphon:evolbis}
%\end{figure}

\begin{figure}
\centering
\includegraphicsmaybe{[width=0.95\textwidth]}{../img/velocity_50.png}
\includegraphicsmaybe{[width=0.95\textwidth]}{../img/velocity_250.png}
\includegraphicsmaybe{[width=0.95\textwidth]}{../img/velocity_500.png}
 \caption{Evolution of velocity field in time in both tanks.}\label{fig:siphon:evol_vel}
\end{figure}
