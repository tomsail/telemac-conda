\chapter{Flow in a channel with slope and friction (friction)}

\section{Description}

This example shows that \telemac{2D} is able to simulate a flow in a channel
with mild slope and friction.

The configuration is a straight channel 410~km long and 450~m wide
with a mild slope ($z = -1.8536585365.10^{-5}x-4$),
see Figure \ref{t2d:Friction:fig:bottom}.

\begin{figure}[!htbp]
 \centering
 \includegraphicsmaybe{[width=\textwidth]}{../img/Bottom.png}
 \caption{Bottom elevation.}
 \label{t2d:Friction:fig:bottom}
\end{figure}

\subsection{Initial and boundary conditions}

The computation is initialised with a constant elevation equal to -9~m
and no velocity.

The boundary conditions are:
\begin{itemize}
\item For the solid walls, a slip condition on channel banks is used for the
velocities,
\item On the bottom, a Strickler law with friction coefficient equal to
70~m$^{1/3}$/s is prescribed,
\item Upstream a flowrate equal to 6,000~m$^3$/s is prescribed,
\item Downstream the water level is suggested to be equal to 0~m.
\end{itemize}

Boundary conditions types can be seen in Figure \ref{t2d:Friction:fig:BC}.

\begin{figure}[!htbp]
 \centering
 \includegraphicsmaybe{[width=\textwidth]}{../img/BC.png}
 \caption{Boundary conditions types.}
 \label{t2d:Friction:fig:BC}
\end{figure}

\subsection{Mesh and numerical parameters}

The mesh is made of 10,524 triangular elements (7,317 nodes),
see zooms around inlet and outlet of the mesh in Figures
\ref{t2d:Friction:fig:meshin} and \ref{t2d:Friction:fig:meshout}.

\begin{figure}[!htbp]
 \centering
 \includegraphicsmaybe{[width=\textwidth]}{../img/Meshin.png}
 \caption{Zoom of the mesh at the inlet.}
 \label{t2d:Friction:fig:meshin}
\end{figure}

\begin{figure}[!htbp]
 \centering
 \includegraphicsmaybe{[width=\textwidth]}{../img/Meshout.png}
 \caption{Zoom of the mesh at the outlet.}
 \label{t2d:Friction:fig:meshout}
\end{figure}

The simulated period is 55,000~s (around 15~h~17~min).

To solve the advection, the HLLC finite volume scheme is used with a desired Courant number equal to 0.8.

\subsection{Physical parameters}

No diffusion is chosen for this computation (constant horizontal viscosity for
velocity equal to 0.~m$^2$/s).

\section{Results}

The flow establishes a steady flow, see Figures \ref{t2d:Friction:freesurf} and
\ref{t2d:Friction:velovect}.

\begin{figure}[H]
\centering
\includegraphicsmaybe{[width=0.9\textwidth]}{../img/FreeSurface.png}
\caption{Free surface elevation at final time step.}
\label{t2d:Friction:freesurf}
\end{figure}

\begin{figure}[H]
\centering
\includegraphicsmaybe{[width=0.9\textwidth]}{../img/Velocity.png}
\caption{Magnitude of velocity at final time step with vectors.}
\label{t2d:Friction:velovect}
\end{figure}

% Comparison with the analytical solution to be done

\section{Conclusions}

This example shows that \telemac{2D} is able to simulate a flow
in a channel with mild slope and friction.
