\chapter{Seiche}

\section{Purpose}
Seiches are resonant oscillations, or ‘normal modes’, of lakes and coastal waters;
that is, they are standing waves with unique frequencies, imposed by the
dimensions of the basins in which they occur. Water moves back and forth across
the basin in a periodic oscillation, alternately raising and lowering sea level
at the basin sides.
Sea level pivots are ‘node’ at which the sea level never changes.
Currents are maximum beneath the node when the sea level is horizontal, and they
vanish when the sea level is at its extremes.

The purpose of this test case is to show that \telemac{2D} compares well against
the analytical solution of the linearised shallow water equations, in a physical
domain close to the linearising assumptions.

At this stage, only the results of the \tel system are presented.
The comparison against the analytical solution will be added later.

\section{Description}
For illustrative purposes we consider a rectangle domain, or flume, 200~m long
and 1.8~m wide – for all intent and purposes, this test case can be considered
as a 1D test case.

The channel bottom is placed at $z$ = 0~m.

\begin{figure}[H]
\centering
\includegraphicsmaybe{[width=0.9\textwidth]}{../img/Mesh.png}
\caption{Mesh of the domain.}
\label{fig:seiche:mesh}
\end{figure}

The edge length of the mesh is uniform, set at about 0.3~m.
It was built as a triangulation of a regular mesh of cell size of 0.25~m.

\subsection{Initial conditions}
The initial free surface elevation is set through the subroutine
\telfile{USER\_CONDIN\_H} as follows:
\begin{equation}
H = H_o + A \cos( \frac{2 \pi X}{L} ),
\end{equation}

where $L$ is the length of the flume (200~m), $H_o$ is the water depth at the
equilibrium (10~m) and $A$ is the initial seiche amplitude (chosen as 0.01~m,
which is small compared to $H_o$).

\section{Reference}
Analytical solution of the linearised shallow water equation\ldots

\section{Results}
%The figure below shows a time series of the free surface elevation at the centre of the flume.
Figure \ref{fig:seiche:result} shows the free surface at the end of the
simulation.

\begin{figure}[H]
\centering
\includegraphicsmaybe{[width=0.9\textwidth]}{../img/free.png}
\caption{Free surface at time 300~s.}
\label{fig:seiche:result}
\end{figure}

\section{Conclusion}
Without a comparison against an analytical solution, all we have demonstrated at
this stage is that \telemac{2D} conserves energy, as the water keeps sloshing
without attenuations of the amplitude of the seiche – and this is already a
interesting result.
