\chapter{Non-Newtonian model (nn\_newt)}
\label{chapter:nnnewt}

\section{Purpose}

The purpose of this test case is to validate the implementation of the
non-Newtonian module.
This test case and documentation is taken from \cite{Ligier2020}.

\section{Theoritical background}

The non-Newtonian fluid behaviour is modeled by an additional friction source
term in the fluid.
This term is treated with a semi-implicit approach in the finite volume framework.

\subsection{Bingham model}

With $\tau$ the viscous shear stress of the fluid, the Bingham model reads:

\begin{equation}
\tau=\left\{
\begin{array}{l}
0 \text{ if } \tau<\tau_y \\
\tau_y+\mu\dot{\gamma} \text{ else}
\end{array}
\right.,
\label{eq:bingham}
\end{equation}

where $\mu$ is the fluid’s dynamic viscosity [Pa$\cdot$s], $\dot{\gamma}$ is the
shear rate [s$^{-1}$] and $\tau_y$ is the fluid’s yield stress [Pa].
$\dot{\gamma}$ can be expressed as:

\begin{equation}
\dot{\gamma}=\frac{KU}{8h},
\end{equation}

with $U$ the depth-averaged flow velocity [m/s], $h$ the water depth [m] and $K$
a resistance parameter for laminar flow [-].
Its value lies in the range 24-108 for smooth surfaces (concrete, asphalt) but
can increase significantly with irregular geometry and roughness (highest values
of approximatively 50,000). When modelling turbulent flows, the lowest and
default value of 24 is recommended \cite{Ligier2020}.

Even though the Bingham model’s mathematical expression is relatively simple,
the discontinuity generated by the yield stress parameter at very low shear
rates is a disadvantage and can lead to numerical instabilities.
To solve this issue, several solution methods have been proposed in the
literature, aiming at replacing the discontinuity by a continuous relationship
between shear stress and shear rate.

\subsubsection{Exponential regularization}

This method is based on the exponential regularization method proposed by
\cite{papanastasiou1987flows}.
An exponential term is added to the yield stress parameter making it possible to
introduce a continuous relationship for low shear rates:

\begin{equation}
  \tau = \tau_y(1-e^{-m\dot{\gamma}})+\mu\dot{\gamma},
\end{equation}

with $m$ a so-called regularization parameter [s] used to control the exponential
growth of shear stress for low shear rates and set to 1,000 in the model.

\subsubsection{Effective viscosity}

This method is based on the effective viscosity concept by rewriting the Bingham
model as:

\begin{equation}
\tau= \mu_{eff}\dot{\gamma},
\end{equation}

with $\mu_{eff}$ the effective viscosity [Pa$\cdot$s], defined by:

\begin{equation}
\mu_{eff}=\frac{\mu_0+\mu K_B\dot{\gamma}}{1+K_B\dot{\gamma}},
\end{equation}

with $K_B=\frac{\mu_0}{\tau_y}$ and $\mu_0=10^3\mu$.

\subsubsection{Cubic equation}

This method is based on a cubic equation for the non-Newtonian shear stress that
is obtained from the integration of the classical Bingham equation for laminar
flow in a wide open channel, and then solving for the depth-averaged flow
velocity as proposed by \cite{rickenmann1990bedload}.
The resulting cubic equation reads:

\begin{equation}
2\tau^3-3\tau^2\left(\tau_y+2\mu \frac{U}{h}\right)+\tau_y^3=0.
\end{equation}

It is solved by keeping the positive root closest from the theoretical value of
$\tau$ defined by \eqref{eq:bingham}.

\subsection{Herschel-Bulkley model}

The equation of the Herschel-Bulkley model reads:

\begin{equation}
\tau=\left\{
\begin{array}{l}
0 \text{ if } \tau<\tau_y \\
\tau_y+K_{HB}\dot{\gamma}^n \text{ else}
\end{array}
\right..
\end{equation}

The Herschel-Bulkley model has been implemented using the exponential
regularization method also used for the Bingham model:

\begin{equation}
  \tau= \tau_y(1-e^{-m\dot{\gamma}})+K_{HB}\dot{\gamma}^n.
\end{equation}

\subsection{Pseudo biphasic model}

The principle of this formulation is to determine the fluid density and the
rheological parameters (yield stress and dynamic viscosity) from the local
sediment volumetric concentration $C_v$ [-].
The type of fluid can therefore be defined by the local sediment volumetric
concentration $C_v$.
This concentration is to be defined by the user in a passive tracer (with
initial and boundary conditions).
The local fluid's bulk density is then computed as:

\begin{equation}
  \rho = \rho_w+(\rho_s-\rho_w)C_v,
\end{equation}

with $\rho_w$ the water density [kg/m$^3$] and $\rho_s$ the sediment specific
density (grains) [kg/m$^3$].

Empirical relationships have been proposed to express the yield stress and the
dynamic viscosity as functions of the sediment volumetric concentration $C_v$
\cite{julien2010erosion}:

\begin{equation}
  \tau_y=a10^{bC_v},
  \label{mix1}
\end{equation}

\begin{equation}
  \mu=c10^{dC_v}.
  \label{mix2}
\end{equation}

The values of the coefficients $a$, $b$, $c$ and $d$ are mainly function of the
nature and percentage of fine particles in the mixture.
Experimental values have been proposed in the literature \cite{julien2010erosion}.

\section{Dambreak test case}

\subsection{Description}

The one-dimensional dam break case presented in \cite{hungr1995model} is used to
test the implemented non-Newtonian models for such applications.
The dam break case corresponds to the instantaneous release of a non-Newtonian
fluid.

\subsection{Theoretical solution}

The theoretical solution of this case provided by \cite{naef2006comparison} and
based on the assumption that the flow profile after reaching equilibrium is
parabolic, gives a front location of $x$ = 1,896~m counted from $x$ = 0~m,
which corresponds to a runout distance of 1,591 m counted from the dam location
($x$ = 305~m).

\subsection{Geometry and mesh}

This case was simulated using a two-dimensional triangular mesh with an element
size of 3~m in both $x$ and $y$ directions.
The model is 3,000~m long in the $x$ direction and 12~m wide in the $y$ direction.
Figure \ref{nnnewt:mesh} presents the mesh used and the domain.

\begin{figure}[!htbp]
 \centering
 \includegraphicsmaybe{[width=\textwidth]}{../img/mesh.png}
 \caption{Mesh of the channel.}
 \label{nnnewt:mesh}
\end{figure}

\subsection{Initial and boundary conditions}

The initial condition is defined by a volume of 305~m in length
(in the $x$ direction) and 30.5~m in height on a flat and dry bottom.
Figure \ref{nnnewt:init} shows the mixture depth at the initial time.

\begin{figure}[!htbp]
 \centering
 \includegraphicsmaybe{[width=\textwidth]}{../img/init.png}
 \caption{Initial mixture depth.}
 \label{nnnewt:init}
\end{figure}

\subsection{Physical parameters}

The fluid properties are a fluid density $\rho$ = 1,835~kg/m$^3$, a yield stress
$\tau_y$ = 1,500~Pa and a dynamic viscosity $\mu$ = 100~Pa$\cdot$s.
Bottom friction has been simulated with a Strickler coefficient of
70~m$^{1/3}$/s to mimic the smooth and plane bottom.

\subsection{Numerical parameters}

The simulations have been performed with the Bingham model for the three options
implemented and the Herschel-bulkley model using the HLLC finite volume scheme
(\telkey{FINITE VOLUME SCHEME} = 5).

\subsection{Results}

The results are presented in Figure \ref{nnnewt:profilebingham} as flow profiles
extracted along the longitudinal axis (defined by $y$ = 6~m) once the fluid has
reached a pseudo-equilibrium state, which occurs after approximatively three minutes.
The flow profiles corresponding to the three different Bingham options
implemented are compared with the analytical solution from
\cite{naef2006comparison}.

\begin{figure}[!htbp]
 \centering
 \includegraphicsmaybe{[width=\textwidth]}{../img/profile-bingham.png}
 \caption{Flow profile after 180 s for the three Bingham options.}
 \label{nnnewt:profilebingham}
\end{figure}

This dam break case has also been simulated with the Herschel-Bulkley model in
which the consistency parameter $K_{HB}$ was taken equal to the dynamic
viscosity used in the Bingham simulations.
Two simulations were performed with power-law index values of $n$ = 0.5 and
$n$ = 1.5.
The results are compared in Figure \ref{nnnewt:profilehb} qualitatively with the
Bingham Option 1 for the flow profiles corresponding to the pseudo-equilibrium
state (after 3 minutes of simulation).

\begin{figure}[!htbp]
 \centering
 \includegraphicsmaybe{[width=\textwidth]}{../img/profile-hb.png}
 \caption{Flow profile after 180 s for the first Bingham option and the Herschley-Bulkley model with $n=0.5$ and $n=1.5$.}
 \label{nnnewt:profilehb}
\end{figure}

\section{Pseudo biphasic}

\subsection{Description}

To illustrate the pseudo-biphasic, variable-density formulation, a simple test
case is used.
The computational domain is composed of a mean channel reach and of a side
channel discharging into the main channel with a 90-degree angle.

\subsection{Geometry and mesh}

The main channel is 90~m long and the side channel is 21~m long.
Both channels are 10~m wide.
The bathymetry is defined as a constant level in all the model.
The computational mesh is composed of triangles with an edge side of
approximatively 1~m.
Figure \ref{pseudo-biphasic:mesh} presents the mesh used and the domain.

\begin{figure}[!htbp]
 \centering
 \includegraphicsmaybe{[width=0.3\textwidth]}{../img/mesh-pb.png}
 \caption{Mesh of the channel.}
 \label{pseudo-biphasic:mesh}
\end{figure}

\subsection{Initial and boundary conditions}

Two inflow boundaries are defined at the upstream end of both channels.
One outflow boundary is defined at the downstream end of the main channel with a
flow depth of 1~m.
Inflow of the non-Newtonian fluid is applied at the upstream end of the side
channel by prescribing a discharge of 4~m$^3$/s and a sediment volumetric
concentration $C_v$ of 0.5 through a passive tracer.
At the inflow boundary, the fluid's bulk density is $\rho$ = 2,000~kg/m$^3$.
Inflow of Newtonian fluid is applied at the upstream end of the main channel by
prescribing a discharge of 2~m$^3$/s and a nil sediment volumetric concentration
through a passive tracer.

\subsection{Physical parameters}

The non-Newtonian fluid density is computed by the model based on a sediment
specific density $\rho_s$ = 3,000~kg/m$^3$ and the specified sediment volumetric
concentration $C_v$.
The Newtonian fluid density is set to $\rho_w$ = 1,000~kg/m$^3$.
The non-Newtonian parameters, yield stress and dynamic viscosity, are computed
by the model with power laws \eqref{mix1} and \eqref{mix2} based on the local
sediment volumetric concentration $C_v$ with the following coefficients:
$a$ = 0.025, $b$ = 8.0, $c$ = 0.001, $d$ = 8.0.
For the non-Newtonian fluid defined with $C_v$ = 0.5, those coefficients yield a
yield stress and dynamic viscosity of 250 Pa and 10 Pa$\cdot$s, respectively
while the Newtonian fluid ($C_v$ = 0) is consequently described with a yield
stress and dynamic viscosity of 0.025 Pa and 0.001 Pa$\cdot$s, respectively,
which is a reasonable approximation.

Bottom friction has been simulated with a Strickler coefficient of 70 m$^{1/3}$/s.

\subsection{Numerical parameters}

The simulations have been performed with the first option of the Bingham model
using the HLLC finite volume scheme (\telkey{FINITE VOLUME SCHEME} = 5).

\subsection{Results}

Figure \ref{pseudo-biphasic:results} presents the steady state conditions for
the flow depth, sediment volumic concentration, fluid density, yield stress and
dynamic viscosity.

\begin{figure}[!htbp]
 \centering
 \includegraphicsmaybe{[width=0.3\textwidth]}{../img/depth.png}
 \includegraphicsmaybe{[width=0.3\textwidth]}{../img/concentration.png}
 \includegraphicsmaybe{[width=0.3\textwidth]}{../img/density.png}\\
 \includegraphicsmaybe{[width=0.3\textwidth]}{../img/yield-stress.png}
 \includegraphicsmaybe{[width=0.3\textwidth]}{../img/viscosity.png}
 \caption{Results of the model.}
 \label{pseudo-biphasic:results}
\end{figure}

\section{Conclusion}

The model is able to properly reproduce non-Newtonian Bingham type of flows.
It is also possible to simulate mixture with water with a simplified model.
