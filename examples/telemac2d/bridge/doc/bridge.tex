\chapter{Bridge}
\section{Purpose}

This case is not properly saying a validation test case but more an example to
demonstrate a possible solution for user to manage bridge in \telemac{2D}.

\section{Description}

In this case, the choice is done to represent the bridge as an obstacle to
the flow (like a dike in the river bed) where \telemac{2D} will compute the
overflowing discharge. Nevertheless, as the part of the discharge passing under
the bridge is not neglictible and also to allow the computation of the full
hydrograph, the flow which passes under the bridge is taken in account by an
internal singularity, under the form of a couple of source and sink nodes.

\subsection{Geometry and mesh}

A river reach with its major bed of 1,000~m long is crossed by a road in the
middle.
At the location of the main channel and under the road, there are 2 rectangular
tubes of 2.5~m width and 1.5~m height.

The mesh of the main channel and the road is regular (10 $\times$ 10~m).
The rest of the mesh is computed without any constraint by Blue Kenue.

\begin{itemize}
\item 5,000 triangular elements,
\item 2,626 nodes,
\item Maximum size range: 16.36~m.
\end{itemize}

Figure~\ref{fig:bridge:mesh} shows the mesh of the study.
\begin{figure}
\centering
\includegraphicsmaybe{[width=.8\textwidth]}{../img/Mesh.png}
\caption{Mesh of the study.}\label{fig:bridge:mesh}
\end{figure}

\subsection{Boundaries}

Lateral:

\begin{itemize}
\item Solid walls with slip condition on both lateral parts of the major bed,
\item Imposed discharge upstream (left).
The discharge varies in time: from 0~m$^3$/s at the beginning,
it increases to 20~m$^3$/s and stays at this value between 100~s and 6,100~s.
Then the discharge increases linearly to 120~m$^{3}$/s at 6,600~s and stays at
this value between 6,600~s and 17,600~s.
After that, the discharge decreases to 20~m$^3$/s and stays at this level
for the rest of the simulation,
\item Imposed water level downstream (right). The value of 3~m is imposed.
\end{itemize}

Bottom:

\begin{itemize}
\item Strickler formula with friction coefficient = 20~m$^{1/3}$/s,
\item Regular slope of 1~m along the model (between 1 and 0 for the main channel),
\item The bank level is 5~m upstream and the major bed grow regularly from 5 to
  10~m.
\end{itemize}

Figure~\ref{fig:bridge:bathy} shows the bottom elevation.
\begin{figure}
\centering
\includegraphicsmaybe{[width=.8\textwidth]}{../img/Bathy.png}
\caption{Topography of the model.}\label{fig:bridge:bathy}
\end{figure}

\subsection{Physical Parameters}

Turbulence: Model of constant viscosity with velocity diffusivity = 1~m$^2$/s.

\subsection{Numerical Parameters}

Type of advection:
\begin{itemize}
\item edge-based N-scheme on velocities (scheme \#14,)
\item conservative + modified SUPG on depth (mandatory),
\item edge-based N-scheme on tracer (scheme \#14).
\end{itemize}

Type of element:
\begin{itemize}
\item  Linear triangle P1 for $h$ and velocities.
\end{itemize}

Solver information:
\begin{itemize}
\item Conjugate gradient solver,
\item Accuracy = 10$^{-8}$.
\end{itemize}

Tracer:

\begin{itemize}
\item Initial concentrations : no tracer in the domain,
\item Conjugate gradient solver (default value),
\item Accuracy = $10^{-10}$.
\end{itemize}

Time data:

\begin{itemize}
\item Time step = 1~s.
\item Simulation duration = 45,000~s.
\end{itemize}

\section{Results}

Figure~\ref{fig:bridge:profile} illustrates the results on a longitudinal
profile located in the middle of the main channel at 3 significant time-steps
\begin{figure}
\centering
\includegraphicsmaybe{[width=.8\textwidth]}{../img/FreeSurface_long_profile.png}
\caption{Free Surface profile along the channel.}\label{fig:bridge:profile}
\end{figure}

During the first part of the simulation (before 6,100~s), the flow stays in the
minor bed as the culverts have sufficient capacities.
Velocities are regular except locally at the location of culvert extremities.
The tracer tends to an equilibrium in all the minor bed
(see Figure \ref{fig:bridge:surf0}).

With the increase of the discharge, the culverts reach a saturation and then the
level upstream increases until the overflow appears and a new equilibrium exists.
This is illustrated by results at 17,400~s (see Figure \ref{fig:bridge:surf1}).
 
When the discharge decreases, there is a progressive diminution of the water
lever upstream the dike.
The simulation recover an equilibrium similar to the first one observed
(see Figure \ref{fig:bridge:surf2}).

\section{Conclusions}

Bridge in \telemac{2D} could be taken in account by the combination of a local
dike (which allows to represent the overflowing part) and some hydraulic
structures which link the upstream and downstream of the dike.

\section{Figures}

\begin{figure}
\centering
 \includegraphicsmaybe{[width=1.\textwidth]}{../img/FreeSurface_0.png}
 \includegraphicsmaybe{[width=1.\textwidth]}{../img/Velo_0.png}
 \includegraphicsmaybe{[width=1.\textwidth]}{../img/Tracer_0.png}
 \caption{Map of the free surface elevation, the velocity and the tracer at $t$ = 6,000~s.}\label{fig:bridge:surf0}
\end{figure}

\begin{figure}
\centering
 \includegraphicsmaybe{[width=1.\textwidth]}{../img/FreeSurface_1.png}
 \includegraphicsmaybe{[width=1.\textwidth]}{../img/Velo_1.png}
 \includegraphicsmaybe{[width=1.\textwidth]}{../img/Tracer_1.png}
 \caption{Map of the free surface elevation, the velocity and the tracer at $t$ = 17,400~s.}\label{fig:bridge:surf1}
\end{figure}

\begin{figure}
\centering
 \includegraphicsmaybe{[width=1.\textwidth]}{../img/FreeSurface_2.png}
 \includegraphicsmaybe{[width=1.\textwidth]}{../img/Velo_2.png}
 \includegraphicsmaybe{[width=1.\textwidth]}{../img/Tracer_2.png}
 \caption{Map of the free surface elevation, the velocity and the tracer at $t$ = 45,000~s.}\label{fig:bridge:surf2}
\end{figure}
