\chapter{Drag force in a channel (dragforce)}

\section{Description}

This example checks that \telemac{2D} is able to represent a drag force
and porosity.

The configuration is a straight channel 200~m long and 40~m wide
with a flat horizontal bottom without slope.

\subsection{Initial and boundary conditions}

The computation is initialised with a constant elevation equal to 4~m
and a constant velocity along $x$ equal to 0.625~m/s.

The boundary conditions are:
\begin{itemize}
\item For the solid walls, a slip condition on channel banks is used for the
velocities,
\item On the bottom, a Strickler law with friction coefficient equal to
40~m$^{1/3}$/s is prescribed,
\item Upstream a flowrate equal to 100~m$^3$/s is prescribed,
\item Downstream the water level is equal to 4~m.
\end{itemize}

\subsection{Mesh and numerical parameters}

The mesh (Figure \ref{t2d:dragforce:fig:meshH})
is made of 7,334 triangular elements (3,740 nodes).
Is is particularly refined around where the drag force and porosity are applied.
%(for -15~m < $x$ < 20~m and -10~m < $y$ < 10~m).

\begin{figure}[!htbp]
 \centering
 \includegraphicsmaybe{[width=\textwidth]}{../img/Mesh.png}
 \caption{Horizontal mesh.}
 \label{t2d:dragforce:fig:meshH}
\end{figure}

The time step is 0.2~s for a simulated period of 2,000~s.

To solve the advection, the method of characteristics
is used for the velocities (scheme 1).
The conjugate gradient
is used for solving the propagation step (option 1) and
the implicitation coefficients
for depth and velocities are respectively equal to 1 and 0.6.

\subsection{Physical parameters}

The $k-\epsilon$ model is used for turbulence modelling.
The drag force is activated by setting the keyword
\telkey{VERTICAL STRUCTURES} = YES.
The default treatment implemented in the \telfile{DRAGFO} subroutine
is used in this example but it can be changed if needed.

In addition, porosity is applied by setting
\telkey{TIDAL FLATS} = YES (default value) +
\telkey{OPTION FOR THE TREATMENT OF TIDAL FLATS} = 3 and
by implementing the user subroutine \telfile{USER\_CORPOR}:

\begin{equation*}
\left\{
    \begin{array}{rl}
        \textrm{if } -10 \le x \le 10, & \textrm{porosity} = \frac{19}{20}, \\
        \textrm{if } x < -10, & \textrm{porosity} = \frac{19}{20} - \frac{1}{20} \frac{x+10}{10},\\
        \textrm{if } x > 10,  & \textrm{porosity} = \frac{19}{20} + \frac{1}{20} \frac{x-10}{10}.
    \end{array}
\right.
\end{equation*}

\section{Results}

The flow establishes a steady flow where the free surface is lighly higher
at the entrance and lightly drops (less than 1~cm) after the area where the drag
force and porosity are applied (see Figure \ref{t2d:dragforce:results}).
The flow accelerates where porosity and drag force are applied and then
retrieves quite similar velocity after.
%, but they are
%lightly greater in the wake of where the drag force is defined,
%compared to along the walls.

\begin{figure}[H]
  \centering
   \subfloat[][Free surface]{
  \includegraphicsmaybe{[width=0.9\textwidth]}{../img/FreeSurface.png}}\\
  \subfloat[][Velocity]{
  \includegraphicsmaybe{[width=0.9\textwidth]}{../img/Velocity.png}}
  \caption{Results.}\label{t2d:dragforce:results}
\end{figure}

\section{Conclusions}

\telemac{2D} is capable to model drag forces and porosity.
