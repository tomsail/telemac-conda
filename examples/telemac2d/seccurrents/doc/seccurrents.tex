\chapter{Secondary flow correction (seccurrents)}

\section{Purpose}

The secondary flow correction proposed by Bernard and Schneider
~\cite{bernard1992, finnie1999} and implemented in \telemac{2D} has been tested
on experimental data from the Riprap Test Facility conducted at the Waterway
Experiment Station of the U.S. Army Engineer Waterways Experimental Station
~\cite{bernard1992}.

\section{Description}

The channel presents four bends and two reversals in curvature, with $L$ = 274~m
long and $B$ = 3.63~m wide with a bed slope of 2.16 per thousand
%\\textperthousand
and 2H:1V bank side slopes (Figure \ref{t2d:seccurrents:fig:Bottom}).

\begin{figure}[!htbp]
 \centering
 \includegraphicsmaybe{[width=\textwidth]}{../img/Bottom.png}
 \caption{Bottom elevation.}
 \label{t2d:seccurrents:fig:Bottom}
\end{figure}

\subsection{Boundary Conditions}

Numerical simulations were performed with a constant discharge $q_{in}$ = 4.2475~m$^3$/s
at inlet and mean flow depth $h_{out}$ = 0.9~m at outlet.

The channel bed has been treated as rigid, with the friction coefficient $C_f$
specified with the Manning relation with a roughness coefficient $n$ = 0.024~s/m$^{-1/3}$.

\subsection{Mesh and numerical/physical parameters}

The computational domain has been discretized with a non-structured triangular
finite element mesh with a total of 25,577 elements and 13,340 nodes,
with mean element size of 0.4~m (Figure \ref{t2d:seccurrents:fig:mesh}).
The discretization of the banks was done with 5 elements on each side of the channel.

\begin{figure}[!htbp]
 \centering
 \includegraphicsmaybe{[width=\textwidth]}{../img/Mesh.png}
 \caption{Horizontal mesh.}
 \label{t2d:seccurrents:fig:mesh}
\end{figure}

The numerical experience was run for $\approx$~5~min until the equilibrium stage
was reached, with a time step $\Delta t$ = 0.1~s.

The secondary flow correction coefficients were set to $A_s$ = 7.071 and $D_s$ = 0.5.

Keyworks and printout variables used for this test case are given below.

\subsubsection{Keywords}
\begin{itemize}
\item \telkey{SECONDARY CURRENTS} = YES,
\item \telkey{PRODUCTION COEFFICIENT FOR SECONDARY CURRENTS} = 7.071,
\item \telkey{DISSIPATION COEFFICIENT FOR SECONDARY CURRENTS} = 0.5.
\end{itemize}

\subsubsection{Printout variables}
\begin{itemize}
\item \texttt{1/R}: The reverse of local radius: $1/r_{sec}$,
\item \texttt{OMEGA}: Flow vorticity: $\Omega$,
\item \texttt{TAU\_S}: Streamwise stresses: $\tau_s$.
\end{itemize}

\section{Results}

Free surface elevation and velocity magnitude at the end of the simulation
can be seen in Figures \ref{fig:seccurrents:FreeSurface} and
\ref{fig:seccurrents:Velocity}.

\begin{figure}[H]
 \centering
 \includegraphicsmaybe{[width=\textwidth]}{../img/FreeSurface.png}
 \caption{Free surface elevation at final time.}
 \label{fig:seccurrents:FreeSurface}
\end{figure}

\begin{figure}[H]
 \centering
 \includegraphicsmaybe{[width=\textwidth]}{../img/Velocity.png}
 \caption{Velocity magnitude at final time.}
 \label{fig:seccurrents:Velocity}
\end{figure}
