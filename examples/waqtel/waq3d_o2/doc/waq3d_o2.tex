\chapter{waq3d\_o2}

\section{Purpose}

This case is an example of the use of the O$_2$ module of \waqtel coupled with
\telemac{3D}.

\section{Description}

A square basin at rest is considered (length and width = 10~m)
with flat bathymetry and elevation at 0~m.

\section{Computational options}

\subsection{Mesh}

The triangular mesh is composed of 272 triangular elements and 159 nodes
(see Figure \ref{fig:waq3d_o2:mesh}).

\begin{figure}[H]
 \centering
 \includegraphicsmaybe{[width=0.6\textwidth]}{../img/res_mesh.png}
\caption{Mesh}
 \label{fig:waq3d_o2:mesh}
\end{figure}

To build the 3D mesh of prisms, 5 planes are regularly spaced over the vertical.
The vertical mesh between nodes of coordinates (-5 ; 0) to (5 ; 0) can be
seen on Figure \ref{fig:waq3d_o2:mesh_section}.

\begin{figure}[H]
 \centering
 \includegraphicsmaybe{[width=0.95\textwidth]}{../img/res_mesh_section.png}
\caption{Vertical mesh}
 \label{fig:waq3d_o2:mesh_section}
\end{figure}

5 (at least) are needed to see variations of dissolved O$_2$ over the vertical.

\subsection{Physical parameters}

The O$_2$ module is activated by setting \telkey{WATER QUALITY PROCESS} = 2
in the \telemac{3D} \telkey{STEERING FILE}.\\

3 tracers are considered:
\begin{itemize}
\item dissolved O$_2$,
\item NH$_4$ load,
\item organic load.
\end{itemize}

Only the following water quality parameters have been changed
in the \waqtel \telkey{STEERING FILE} compared to the default values:
\begin{itemize}
\item \telkey{K2 REAERATION COEFFICIENT} = 0.3,
\item \telkey{FORMULA FOR COMPUTING K2} = 0 (i.e. $k_2$ is constant),
\item \telkey{O2 SATURATION DENSITY OF WATER (CS)} = 9~mgO$_2$/l ($C_s$ is constant),
\item \telkey{WATER TEMPERATURE} = 20$^\circ$C (which is the mean temperature of water).
\end{itemize}

\subsection{Initial and Boundary Conditions}

The initial water depth is 2~m with a fluid at rest.\\
%
The initial values for tracers are homogeneous:\\
([O$_2$], [L], [NH$_4$]) =
(50~mgO$_2$/l, 3~mgO$_2$/l, 0.01~mgNH$_4$/l).\\
%
There are only closed lateral boundaries with free slip condition and no
friction at the bottom.

\subsection{General parameters}

The time step is 4~s for a simulated period of 40,000~s.

\subsection{Numerical parameters}

Basin at rest (no advection nor diffusion for hydrodynamics).\\

2 new default values since release 8.1 for the advection and diffusion of the
tracers are taken in the \telemac{3D} \telkey{STEERING FILE} :
\begin{itemize}
\item \telkey{ACCURACY FOR DIFFUSION OF TRACERS} = 10$^{-8}$,
\item \telkey{SCHEME OPTION FOR ADVECTION OF TRACERS} = 4
  (combined with \telkey{SCHEME FOR ADVECTION OF TRACERS} = 5 leads to LIPS use).
\end{itemize}

\section{Results}

Figure \ref{fig:waq3d_o2:res} shows the 3 tracers evolution along time.

\begin{figure} [H]
\centering
\includegraphicsmaybe{[width=\textwidth]}{../img/res_tracers.png}
 \caption{Tracers evolution for different tracers of the O$_2$ module}
 \label{fig:waq3d_o2:res}
\end{figure}

\section{Conclusion}

The O$_2$ module of \waqtel can be coupled with \telemac{3D}.
