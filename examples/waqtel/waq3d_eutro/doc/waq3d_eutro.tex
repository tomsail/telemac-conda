\chapter{waq3d\_eutro}

\section{Purpose}

This case is an example of the use of the EUTRO module of \waqtel coupled with
\telemac{3D}.

\section{Description}

A square basin at rest is considered (length and width = 10~m)
with flat bathymetry and elevation at 0~m.

\section{Computational options}

\subsection{Mesh}

The triangular mesh is composed of 272 triangular elements and 159 nodes
(see Figure \ref{fig:waq3d_eutro:mesh}).

\begin{figure}[H]
 \centering
 \includegraphicsmaybe{[width=0.6\textwidth]}{../img/res_mesh.png}
\caption{Mesh}
 \label{fig:waq3d_eutro:mesh}
\end{figure}

To build the 3D mesh of prisms, 5 planes are regularly spaced over the vertical.
The vertical mesh between nodes of coordinates (-5 ; 0) to (5 ; 0) can be
seen on Figure \ref{fig:waq3d_eutro:mesh_section}.

\begin{figure}[H]
 \centering
 \includegraphicsmaybe{[width=0.95\textwidth]}{../img/res_mesh_section.png}
\caption{Vertical mesh}
 \label{fig:waq3d_eutro:mesh_section}
\end{figure}

\subsection{Physical parameters}

The EUTRO module is activated by setting \telkey{WATER QUALITY PROCESS} = 5
in the \telemac{3D} \telkey{STEERING FILE}.\\

8 tracers are considered:
\begin{itemize}
\item phytoplanktonic biomass (PHY),
\item dissolved mineral phosphorus (PO$_4$),
\item degradable phosphorus non assimilated by phytoplankton (POR),
\item dissolved mineral nitrogen assimilated by phytoplankton (NO$_3$),
\item degradable nitrogen assimilated by phyto phytoplankton (NOR),
\item ammoniacal load (NH$_4$),
\item organic load (L),
\item dissolved oxygen (O$_2$).
\end{itemize}

Only the following water quality parameters have been changed
in the \waqtel \telkey{STEERING FILE} compared to the default values:
\begin{itemize}
\item \telkey{K2 REAERATION COEFFICIENT} = 0.3,
\item \telkey{FORMULA FOR COMPUTING K2} = 0 (i.e. $k_2$ is constant),
\item \telkey{O2 SATURATION DENSITY OF WATER (CS)} = 9~mgO$_2$/l ($C_s$ is constant),
\item \telkey{SECCHI DEPTH} = 0.1~m,
\item \telkey{VEGETAL TURBIDITY COEFFICIENT WITHOUT PHYTO} = 0.01~m$^{-1}$,
\item \telkey{SUNSHINE FLUX DENSITY ON WATER SURFACE} = 0.01~W/m$^2$,
\item \telkey{WATER TEMPERATURE} = 20$^\circ$C (which is the mean temperature of water).
\end{itemize}

\subsection{Initial and Boundary Conditions}

The initial water depth is 2~m with a fluid at rest.\\
%
The initial values for tracers are homogeneous:\\
([PHY],[PO4],[POR],[NO3],[NOR],[NH$_4$],[L],[O$_2$]) =
(50~$\mu$gChlA/l, 3~mg/l, 0.01~mg/l, 2.9~mg/l, 9~mg/l, 1~mgNH$_4$/l, 0.2~mgO$_2$/l, 0.06~mgO$_2$/l).\\
%
There are only closed lateral boundaries with free slip condition and no
friction at the bottom.

\subsection{General parameters}

The time step is 1~h = 3,600~s for a simulated period of 8,760~h = 365~days.

\subsection{Numerical parameters}

Basin at rest (no advection nor diffusion for hydrodynamics).\\

The advection scheme for the tracers is LIPS (new default value since release 8.1) activated
with the combo \telkey{SCHEME FOR ADVECTION OF TRACERS} = 5
and \telkey{SCHEME OPTION FOR ADVECTION OF TRACERS} = 4.

Moreover, the \telkey{ACCURACY FOR DIFFUSION OF TRACERS} is to be set to 10$^{-15}$ to
get mass balance with a good accuracy.
For this example, the current default value = 10$^{-8}$ is not enough.

\section{Results}

Figure \ref{fig:waq3d_eutro:res} shows the 8 tracers evolution along time.
All of them rather quickly reach an asymptote.

\begin{figure} [H]
\centering
\includegraphicsmaybe{[width=\textwidth]}{../img/res_tracers.png}
 \caption{Tracers evolution for different tracers of the EUTRO module}
 \label{fig:waq3d_eutro:res}
\end{figure}

\section{Conclusion}

The EUTRO module of \waqtel can be coupled with \telemac{3D}.
