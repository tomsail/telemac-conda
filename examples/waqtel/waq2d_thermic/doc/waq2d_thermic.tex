% case name
\chapter{waq2d\_thermic}
%
% - Purpose & Description:
%     These first two parts give reader short details about the test case,
%     the physical phenomena involved, the geometry and specify how the numerical solution will be validated
%
\section{Purpose}
%
This case is an example of the use of the heat exchange with atmosphere module
of \waqtel (called THERMIC) coupled with \telemac{2D}.
%
\section{Description}
%
A square basin at rest is considered (length and width = 10~m)
with flat bathymetry and elevation at 0~m.
%
\section{Computational options}
%
\subsection{Mesh}
%
The triangular mesh is composed of 272 triangular elements and 159 nodes
(see Figure \ref{fig:waq2d_thermic:mesh}).

\begin{figure}[H]
 \centering
 \includegraphicsmaybe{[width=0.6\textwidth]}{../img/res_mesh.png}
\caption{Mesh}
 \label{fig:waq2d_thermic:mesh}
\end{figure}
%
\subsection{Physical parameters}
%
The heat exchange module is activated by setting \telkey{WATER QUALITY PROCESS}
= 11 in the \telemac{2D} \telkey{STEERING FILE}.\\

Only temperature is considered as a tracer.\\

Several default values are modified in the \waqtel \telkey{STEERING FILE}:
\begin{itemize}
\item \telkey{WATER DENSITY} = 1,000~kg/m$^3$ (default = 999.972~kg/m$^3$),
%\item \telkey{AIR SPECIFIC HEAT} = 1,002~J/kg/$^{\circ}$C (default =
%  1,005~J/kg/$^{\circ}$C),
\item \telkey{COEFFICIENTS OF AERATION FORMULA} = (0.025 ; 0.025) (default = (0.002 ; 0.0012)),
\item \telkey{COEFFICIENT OF CLOUDING RATE} = 0.2 (usual value in 2D, but default = 0.17),
\item \telkey{COEFFICIENTS FOR CALIBRATING ATMOSPHERIC RADIATION} = 0.85 (default = 0.97).
\end{itemize}

One default value is modified in the \telemac{2D} \telkey{STEERING FILE}
(\telkey{AIR PRESSURE} = YES)
%\telkey{DIFFUSION OF TRACERS} = NO.
whereas the keyword \telkey{ATMOSPHERE-WATER EXCHANGE MODEL} = 0 must not be
changed in 2D!
%
\subsection{Initial and Boundary Conditions}
%
The initial water depth is 1~m with a fluid at rest.\\
The initial temperature is 17~$^{\circ}$C in the whole domain.\\

There are only closed lateral boundaries with free slip condition and no
friction at the bottom.
%
\subsection{General parameters}
%
The time step is 10~min = 600~s for a simulated period of 30~days
(= 2,592,000~s).
%
\subsection{Numerical parameters}
%
Basin at rest (no advection nor diffusion).
No diffusion neither for hydrodynamics or tracers.
%
% - Results:
%     We comment in this part the numerical results against the reference ones,
%     giving understanding keys and making assumptions when necessary.
%
%
\section{Results}
%
Figure \ref{fig:waq2d_thermic:res} shows the temperature evolution along time.

\begin{figure} [H]
\centering
\includegraphicsmaybe{[width=\textwidth]}{../img/res_temp.png}
 \caption{Temperature evolution}
 \label{fig:waq2d_thermic:res}
\end{figure}
%
\section{Conclusion}
%
\waqtel is able to model heat exchange with atmosphere phenomena when coupled
with \telemac{2D}.
%
% Here is an example of how to include the graph generated by validate_telemac.py
% They should be in test_case/img
%\begin{figure} [!h]
%\centering
%\includegraphics[scale=0.3]{../img/mygraph.png}
% \caption{mycaption}\label{mylabel}
%\end{figure}
