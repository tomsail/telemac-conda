\chapter{waq2d\_biomas}

\section{Purpose}

This case is an example of the use of the BIOMASS module of \waqtel coupled with
\telemac{2D}.

\section{Description}

A square basin at rest is considered (length and width = 10~m)
with flat bathymetry and elevation at 0~m.

\section{Computational options}

\subsection{Mesh}

The triangular mesh is composed of 272 triangular elements and 159 nodes
(see Figure \ref{fig:waq2d_biomas:mesh}).

\begin{figure}[H]
 \centering
 \includegraphicsmaybe{[width=0.6\textwidth]}{../img/res_mesh.png}
\caption{Mesh}
 \label{fig:waq2d_biomas:mesh}
\end{figure}

\subsection{Physical parameters}

The BIOMASS module is activated by setting \telkey{WATER QUALITY PROCESS} = 3
in the \telemac{2D} \telkey{STEERING FILE}.\\

5 tracers are considered:
\begin{itemize}
\item phytoplanktonic biomass (PHY),
\item dissolved mineral phosphorus (PO$_4$),
\item degradable phosphorus non assimilated by phytoplankton (POR),
\item dissolved mineral nitrogen assimilated by phytoplankton (NO$_3$),
\item degradable nitrogen assimilated by phyto phytoplankton (NOR),
\end{itemize}

Only the following water quality parameters have been changed
in the \waqtel \telkey{STEERING FILE} compared to the default values:
\begin{itemize}
\item \telkey{SECCHI DEPTH} = 0.1~m,
\item \telkey{VEGETAL TURBIDITY COEFFICIENT WITHOUT PHYTO} = 0.01~m$^{-1}$,
\item \telkey{SUNSHINE FLUX DENSITY ON WATER SURFACE} = 0.01~W/m$^2$,
\item \telkey{WATER TEMPERATURE} = 20$^\circ$C (which is the mean temperature of water).
\end{itemize}

\subsection{Initial and Boundary Conditions}

The initial water depth is 2~m with a fluid at rest.\\
%
The initial values for tracers are homogeneous:\\
([PHY],[PO4],[POR],[NO3],[NOR]) =
(0.5~$\mu$gChlA/l, 3~mg/l, 1.1~mg/l, 0.2~mg/l, 4.9~mg/l).\\
%
There are only closed lateral boundaries with free slip condition and no
friction at the bottom.

\subsection{General parameters}

The time step is 4~s for a simulated period of 86,400~s =~1~day.

\subsection{Numerical parameters}

Basin at rest (no advection nor diffusion).

\section{Results}

Figure \ref{fig:waq2d_biomas:res} shows the 5 tracers evolution along time.

Dissolved PO$_4$, dissolved NO$_3$ and non assimilated NOR are constant in time
whereas phytoplankton PHY and non assimilated POR decrease during the simulation
(less quicker for POR).

\begin{figure} [H]
\centering
\includegraphicsmaybe{[width=\textwidth]}{../img/res_tracers.png}
 \caption{Tracers evolution for different tracers of the BIOMASS module}
 \label{fig:waq2d_biomas:res}
\end{figure}

\section{Conclusion}

The BIOMASS module of \waqtel can be coupled with \telemac{2D}.
