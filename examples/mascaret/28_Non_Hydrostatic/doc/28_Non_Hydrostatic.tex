\chapter{Non\_Hydrostatic}
%

% - Purpose & Problem description:
%     These first two parts give reader short details about the test case,
%     the physical phenomena involved and specify how the numerical solution will be validated

\section{Purpose}
This test shows the capability of the transcritical kernel of \mascaret to deal with non-hydrostatic terms to model the Favre's waves.

These secondary waves appear during a sudden variation of flow in a channel and are in the form of undulations superimposed on the body of the Saint-Venant wave.

%--------------------------------------------------------------------------------------------------
\section{Description}

\subsection{Geometry}

The domain is a rectangular channel of $30\ m$ long and $40\ cm$ width with a slope of $4\%$ .

\subsection{Initial condition}

The steady state corresponding to the normal height $h_n = 0.2 m$

\subsection{Boundary condition}

A constant discharge upstream : $ Q = 0.035\ m^{3}.s^{-1} $

A hydrograph to simulate a flow trigger downstream on a very short time, from $ Q = 0.035\ m^{3}.s^{-1} $ to $ Q = 0\ m^{3}.s^{-1} $ in $ 0.07\ s $


\subsection{Physical parameters}

The roughness coefficient is chosen so that the normal height is $h_n = 0.2\ m$ hence a value $K = 102\ m^{1/3}.s^{-1}$ (application of the Strickler formula).

\subsection{Numerical parameters}

The mesh is define with a space step of $5\ cm$.

The vertical discretization of the Cross Sections is homogeneous and equal to $1\ cm$.

The simulation time is set to $6\ s$.

The time step is variable with a wishes Courant number of $0.8$.

The initial time step is chosen to $0.01\ s$.

%--------------------------------------------------------------------------------------------------
\section{Results}

The evolution in time of the water level with and without take in account the non-hydrostatic terms and the comparison to experimental measurements at $2\ m$ of the downstream limit of the domain is presented on figure \ref{mascaret:nonhydro:evol}.

\begin{figure}[H]
\centering
\includegraphicsmaybe{[width=\textwidth]}{../img/evol.png}
\caption{Water level evolution at $x = 28 m$}
\label{mascaret:nonhydro:evol}
\end{figure}

%--------------------------------------------------------------------------------------------------

\section{Conclusion}
The transcritical kernel of \mascaret is able to model the non hydrostatic term of the Favre's waves which superimpose to the main Saint-Venant wave.

Calculated amplitude is lower and damping is faster compared to experimental data.

A high order scheme tends to solve this problem \cite{Bristeau2011}.