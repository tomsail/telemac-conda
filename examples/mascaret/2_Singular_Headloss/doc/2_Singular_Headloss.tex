\chapter{Singular\_Head\_Loss}
%

% - Purpose & Problem description:
%     These first two parts give reader short details about the test case,
%     the physical phenomena involved and specify how the numerical solution will be validated

\section{Purpose}
The purpose of this test case is to validate the steady computation kernel and the transient kernel in transcritical mode, in the case of a rectangular channel with a singular head loss.
 
%--------------------------------------------------------------------------------------------------
\section{Description}

\subsection{Geometry}

The domain is a rectangular channel of $ 4990\ m $ length and $ 1\ m $ width with nul slope.
The geometry is described by 2 cross section located at $ X = 0\ m $ and $ X = 4990\ m $ .
The local Head Loss is located a the abscissa $ X = 2500 \ m $.

\subsection{Initial condition}

No initial condition for the steady kernel (not requested).

An initial linear varying water level (between $ Z = 1.5\ m $ upstream and $ Z = 1\ m $ downstream) with a constant discharge of $ Q = 1\ m^{3}.s^{-1} $ for the transcritical kernel.

\subsection{Boundary condition}

A constant discharge upstream : $ Q = 1\ m^{3}.s^{-1} $

A constant level downstream : $ Z = 1\ m $


\subsection{Physical parameters}

The roughness is taken in account with a Strickler coefficient value $ K = 90\ m^{1/3}.s^{-1} $

The local Head Loss coefficient is : $0.5$

\subsection{Numerical parameters}

The mesh is define with a space step of $ 10\ m $. The domain is thus devided in 499 elements

The vertical discretization of the Cross Sections is homogeneous and equal to $ 2\ m $.

%No time step parameters for the steady kernel.

%For the transcritical kernel

%The simulation duration is set to $ 2000 $ time step.

%The time step is variable with a wishes Courant number of $ 0.8 $ for the explicit form and a wishes Courant number of $ 2 $ for the implicit form.

%The initial time step is chosen to $ 1\ s $.

%--------------------------------------------------------------------------------------------------
\section{Results}

\subsection{Analytic solution}
With a singular Head Loss, the momentum equation is written

\begin{equation}
 \frac{\partial}{\partial x}\left(\displaystyle \frac{Q^2}{Lh}+\frac{1}{2}gLh^2\right)=-\frac{\partial{}}{\partial x}\displaystyle(\alpha V^{2}_{amont})\times gLh
\end{equation}
with
\begin{description}
\item $ Q $ : 	total discharge ($ m^3.s^{-1} $)
\item $ L $ : 	width of the domain ($ m $)
\item $ h $ : 	water height ($ m $)
\item $ V_{upstream} $: velocity at the upstream cross section of the head loss ($ m.s^{-1} $)
\item $ \alpha $: head loss coefficient
\end{description}

By discretizing this equation with finite difference method, we obtain ($ L = 1\ m $) :

\begin{equation}
\left[\frac{2}{h_{1}+h_{2}}\times \frac{Q^{2}_{1}-Q^{2}_{2}}{\Delta x}\right] +\left[\frac {(Q_{1}+Q_{2})^{2}}{(h_{1}+h_{2})^2}+g \times \frac{h_{1}+h_{2}}{2}\right]\frac{\Delta h }{\Delta x}=-\frac{\alpha}{2\Delta x}\left(\frac{Q_{1}}{h_{1}}\right)^2\times \frac{h_{1}+h_{2}}{2}
\end{equation}

Index $1$ correspond to upstream section, index $2$ correspond to downstream section of the head loss.

Numerical application:
\begin{description}
\item $ Q_{1} = 1\ m^3.s^{-1} $ 
\item $ Q_{2} = 1\ m^3.s^{-1} $
\item $ h_{1} = 1.2453 \ m $
\item $ h_{2} = 1.2272 \ m $
\item $ \Delta x = 10 \ m $
\item $ \alpha = 0.5 $
\end{description}

Those values gives $ \Delta h_{analytique} = 0.0172\ m $

Figure \ref{mascaret:singular_headloss:long} show the comparison of water height with the different kernel.
Both kernel gives similar results.
The computed head loss are:
\begin{description}
\item $ \Delta h_{calculee} $ avec le noyau permanent 7.0 : $ 0.018 \ m $
\item $ \Delta h_{calculee} $ avec le noyau transitoire transcritique : $ 0.018 \ m $
\end{description}
The relative error is : $ 5 \% $.

\begin{figure}[H]
\centering
\includegraphicsmaybe{[width=\textwidth]}{../img/pfl_long.png}
\caption{Comparison of computed water height}
\label{mascaret:singular_headloss:long}
\end{figure}

%--------------------------------------------------------------------------------------------------

\section{Conclusion}
The treatment of local head loss is good.

The differences between the two kernels are nearly 0.
