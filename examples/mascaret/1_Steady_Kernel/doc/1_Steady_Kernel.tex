\chapter{Steady\_State}
%

% - Purpose & Problem description:
%     These first two parts give reader short details about the test case,
%     the physical phenomena involved and specify how the numerical solution will be validated

\section{Purpose}
This test shows the capability of \mascaret kernels to well compute a steady state in the very simple case of a rectangular channel.

A comparison of the result of the steady kernel and the converged results of the transcritical kernel, both implicit and explicit form is also done with the analytical solution.

%--------------------------------------------------------------------------------------------------
\section{Description}

\subsection{Geometry}

The domain is a rectangular channel of $ 10000\ m $ long and $ 100\ m $ width with a slope of $ 0.05\% $.
The geometry is described by 2 cross section located at $ X = 0\ m $ and $ X = 10000\ m $ .

\subsection{Initial condition}

No initial condition for the steady kernel (not requested).

An initial constant water level $ Z = 8\ m $ with a constant discharge of $ Q = 1000\ m^{3}.s^{-1} $ for the other simulation.

\subsection{Boundary condition}

A constant discharge upstream : $ Q = 1000\ m^{3}.s^{-1} $

A constant level downstream : $ H = 3\ m $


\subsection{Physical parameters}

The roughness coefficient is chosen so that the normal height is $ h_n = 5\ m $ hence a value $ K = 30.6\ m^{1/3}.s^{-1} $ (application of the Strickler formula).

\subsection{Numerical parameters}

The mesh is define with a space step of $ 100\ m $.

The vertical discretization of the Cross Sections is homogeneous and equal to $ 1\ m $.

No time step parameters for the steady kernel.

For the transcritical kernel

The simulation duration is set to $ 2000 $ time step.

The time step is variable with a wishes Courant number of $ 0.8 $ for the explicit form and a wishes Courant number of $ 2 $ for the implicit form.

The initial time step is chosen to $ 1\ s $.

%--------------------------------------------------------------------------------------------------
\section{Results}

\subsection{Analytic solution}

In steady state, the Saint-Venant equation system is reduced to the dynamic equation, which is itself simplified. In addition, in a uniform channel, of rectangular geometry, the expression of the friction term is further simplified when this friction is assumed to be zero on the banks: the hydraulic radius is then equal to the water height $ h $, that is ie the variable to be determined. The solution is then written:
\begin{equation}
 \frac{\partial h}{\partial x}=\frac{I-\displaystyle \frac{q^2}{K^2h^{10/3}}}{1-\displaystyle \frac{q^2}{gh^3}}
\end{equation}
with
\begin{description}
\item $ I $: Bottom slope
\item $ K $: Strickler friction coefficient
\item $ q $: Discharge per width unit
\end{description}

This differential equation on $ h $ is solved using a Runge-Kutta method ($ 10\ m $ step, $ 4^{th} $ order)

\begin{table}[H]
\centering
\begin{tabular}{c|c|c|c|c}
\multicolumn{1}{|p{2.5cm}|}{\centering Abscissae}
& \multicolumn{1}{|p{2.5cm}|}{\centering Analytic \\ Solution}
& \multicolumn{1}{|p{2.5cm}|}{\centering Steady \\ Kernel}
& \multicolumn{1}{|p{2.5cm}|}{\centering Transcritical \\ Kernel \\ Explicit}
& \multicolumn{1}{|p{2.5cm}|}{\centering Transcritical \\ Kernel \\ Implicit} \\
\hline
\InputIfFileExists{../img/table.txt}{}{}{}{}{}\\
\end{tabular}
\label{mascaret:steady_kernel:tab1}
\caption{Numerical comparison of analytic and computed water height}
\end{table}

Figures \ref{mascaret:steady_kernel:long} and \ref{mascaret:steady_kernel:zoom} show the comparison of water height with the different kernel.
We could also see in figure \ref{mascaret:steady_kernel:long} that the water height upstream tends to approach asymptotically the normal height $ h_n $ which is theoretically exact

\begin{figure}[H]
\centering
\includegraphicsmaybe{[width=\textwidth]}{../img/long.png}
\caption{Comparison of analytic and computed water height}
\label{mascaret:steady_kernel:long}
\end{figure}

\begin{figure}[H]
\centering
\includegraphicsmaybe{[width=\textwidth]}{../img/zoom.png}
\caption{Zoom of figure \ref{mascaret:steady_kernel:long} between $ 9000\ m $ and $ 10000\ m $}
\label{mascaret:steady_kernel:zoom}
\end{figure}

%--------------------------------------------------------------------------------------------------

\section{Conclusion}
The treatment of gravity-friction terms in the dynamic equation is good.

The comparison of all solutions to the analytical one show the good agreement of all kernels.

