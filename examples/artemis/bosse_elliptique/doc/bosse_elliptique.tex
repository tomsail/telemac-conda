\chapter{Elliptical bumb on beach.: bosse\_elliptique}

\section{Summary}
The objective of this test case is to analyze the behavior of Artemis during
the transformation of the swell over a bump on a beach. When we consider
exclusively the beach, the transformation of the swell is only due to
refraction. In presence of the bump, the coupling between refraction and
diffraction appears. The refraction methods, applied in this case, lead to a
convergence of the swell behind the hump, associated with a concentration of
energy. In reality, the energy of the swell is dispersed, which is mainly
due to diffraction in the lateral direction.

The results are compared with experimental results presented by Berkhoff
\cite{berkhoff1982}. They verify that the coupling between refraction and
diffraction is well taken into account by the Artemis code.


\section{Description of the test case}

The modeled case represents the one realized in \cite{berkhoff1982}, i.e. a
bump, on a beach on the side opposite to the generation of the swell. The
angle between the beach and the swell direction is 20$^\circ$.
The side edges are solid walls. The slope and curvature terms are taken into
account  in the Berkhoff equation (extended diffraction-refraction equation,
option {\it RAPIDLY VARYING TOPOGRAPHY}).

\subsection{Geometry}
Domain
\begin{itemize}
 \item  Size domain: 30 m x 35 m
\item Max depth: 0.45m
\item Beach slope: 1/50
\end{itemize}
  Elliptical Bump
\begin{itemize}
\item   Large axis:  8m
\item Small axis: 6m
\item Max height: 0.2m
\end{itemize}

\subsection{ Mesh}
\begin{itemize}
\item Triangular elements: 49083
\item Mesh size: 20 cm
\item Nodes: 24842
\item Meshes per wavelength: 7.5 (for h = 0.45 m)
\end{itemize}

\subsection{ Boundary conditions}

\begin{figure}[h]
\begin{center}
  \includegraphicsmaybe{[width=0.9\textwidth]}{bc_bosse_elipt.png}
\end{center}
\caption{Dimensions and Boundary conditions}
\label{fig:bc_bosse_elipt}
\end{figure}

Incidental swell:
\begin{itemize}
\item  Period: 1 s
\item  Wave height : 0.0464 m
\item  Zero phase on the whole boundary : ALFAP=0
\item  Direction of propagation: 90$^\circ$.
\end{itemize}
Solid walls:
\begin{itemize}
  \item Reflection coefficient: 0
\item  Phase shift : 0$^\circ$
\item  Angle of attack : 90$^\circ$
\end{itemize}

Liquid wall:
\begin{itemize}
\item  Reflection coefficient: 0
\item  Phase shift: 0$^\circ$
\item  Angle of attack: 20$^\circ$
\end{itemize}

\section{Reference solution}

The experimental results obtained in \cite{berkhoff1982} constitute the reference values,
divided into 8 sections. We compare the wave heights obtained numerically and experimentally by
presenting the amplification coefficient, defined by the following formula:
$$
\frac{H(x,y)}{H_{incident}}
$$

\section{Results}

\begin{figure}[h]
\begin{center}
  \includegraphicsmaybe{[width=0.9\textwidth]}{section_bosse_elipt.png}
\end{center}
\caption{definition of sections}
\label{fig:defsection_bosse_elipt}
\end{figure}

\begin{figure}[h]
\begin{center}
  \includegraphicsmaybe{[width=0.45\textwidth]}{resu_berkhoff.png}
  \includegraphicsmaybe{[width=0.45\textwidth]}{resu_artemis.png}
\end{center}
\caption{ amplification coefficient of wave height with Berkhoff (on left) and artemis (on right)}
\label{fig:bosse_elipt_resu_comp}
\end{figure}

The wave amplitudes obtained with the solver 3 (figure \ref{fig:bosse_elipt_resu_comp}) are compared to
those presented by
Berkhoff \cite{berkhoff1982}. The measurements were taken along different sections parallel to
the x and y axes, represented on the diagram below (figure \ref{fig:defsection_bosse_elipt}).


\begin{figure}[h]
\begin{center}
  \includegraphicsmaybe{[width=0.9\textwidth]}{sections_bosse.png}
\end{center}
\caption{amplification coefficients of wave height for the 8 sections.}
\label{fig:ressections_bosse}
\end{figure}

The comparison (figure \ref{fig:ressections_bosse}) between the results obtained by
Artemis and those obtained in \cite{berkhoff1982} shows good correspondences. We can
clearly see the effect of the bump: a maximum
amplification of 227\%. Experimentally, this maximum is 221\%. This highlights that
Artemis takes into account the refraction-diffraction interactions.


\section{Conclusions}
This test case compares the results produced by Artemis with experimental results on
physical model. It allows to validate the modeling of the combined effects of refraction and
diffraction.
