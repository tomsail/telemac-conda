\chapter{Port of Borme les Mimosas (creocean)}
\section{Purpose}
This Artemis case describes the analysis of the harbor agitation in the port of Borme les Mimosas. It
gives an example of the use of the code in a concrete case: choice of different reflection coefficients
of reflection, parallel calculation, etc...
\section{Description of the problem}
\subsection{Geometry and mesh}
The domain is equally delimited by the solid and liquid walls. Several assumptions
are applied for the liquid boundary conditions to the north, as well as for the solid walls.txt
We will detail them in 2.3. The bathymetry can be visualized in Figure \ref{fig:creocean_bathy}.
The initial coastline of the free surface is 0.5 m.
\begin{figure}[h]
\begin{center}
  \includegraphicsmaybe{[width=0.5\textwidth]}{../img/Bathy.png}
\end{center}
\caption{Geometry.}
\label{fig:creocean_bathy}
\end{figure}

The mesh consists of 62261 triangle elements and 31645 nodes. The size
of a mesh is 3m. The mesh can be visualized on figure \ref{fig:creocean_mesh}.
\begin{figure}[h]
\begin{center}
  \includegraphicsmaybe{[width=0.5\textwidth]}{../img/Mesh.png}
\end{center}
\caption{Mesh.}
\label{fig:creocean_mesh}
\end{figure}
\subsection{Boundary conditions}
\begin{figure}[h]
\begin{center}
  \includegraphicsmaybe{[width=0.5\textwidth]}{creocean_bc.png}
\end{center}
\caption{Bound.}
\label{fig:creocean_bc}
\end{figure}
Incident wave:
Wave 1 :
\begin{itemize}
\item Peak period = 8 s
\item Max and Min period : 12 s and 6 s.
\item Spectrum parameter: gamma = 2.
\item Wave height : 2 m
\item Direction of propagation $\theta =180^\circ$
\item Exit direction : 63$^\circ$.
\end{itemize}
Wave 2:
\begin{itemize}
\item Peak period = 8 s
\item Max and Min period : 12 s and 6 s.
\item Spectrum parameter : gamma = 2.
\item Wave height : 2 m
\item Direction of propagation : $\theta =180^\circ$
\item Exit direction : 0 $^\circ$.
\end{itemize}
Wave 3:
\begin{itemize}
\item Peak period = 8 s
\item Max and Min period : 12 s and 6 s.
\item Spectrum parameter : gamma = 2.
\item Wave height : 2 m
\item Direction of propagation : $\theta =180^\circ$.
\item Exit direction : 73$^\circ$.
\end{itemize}

For these three boundaries, the phase is computed incrementally on the boundary
from a reference point, which we will denote A :
\begin{equation}
  \Phi_I=\Sigma_{P=A}^{I-1}k(P)cos(\Theta) (x_{P+1}-x_{P})
  \end{equation}
Where $k(P)$ is the wave number at the boundary point P.
This phase, converted to degrees, is provided to the code in the ALFAP table.
Walls :
Absorbing wall 1 :
\begin{itemize}
\item Reflection coefficient = 0.
\item Phase shift : 0$^\circ$.
\item Angle of attack: 0$^\circ$.
\end{itemize}
Wall 1 :
\begin{itemize}
\item Reflection coefficient = 0.05
\item Phase shift : 0$^\circ$.
\item Angle of attack : 0$^\circ$.
\end{itemize}
Wall 2 :
\begin{itemize}
\item Reflection coefficient = 0.15
\item Phase shift : 0$^\circ$.
\item Angle of attack : 45$^\circ$.
\end{itemize}
Wall 3 :
\begin{itemize}
\item Reflection coefficient = 0.05
\item Phase shift : 0$^\circ$.
\item Angle of attack : 0$^\circ$.
\end{itemize}
Wall 4 (+ island):
\begin{itemize}
\item Reflection coefficient = 1.
\item Phase shift : 0$^\circ$.
\item Angle of attack : 0$^\circ$..
\end{itemize}
Wall 5 :
\begin{itemize}
\item Reflection coefficient = 0.15
\item Phase shift : 0$^\circ$.
\item Angle of attack : 0$^\circ$.
\end{itemize}
\section{Results}

\begin{figure}[h]
\begin{center}
  \includegraphicsmaybe{[width=0.5\textwidth]}{../img/WaveHeight.png}
\end{center}
\caption{Wave Height.}
\label{fig:creocean_results}
\end{figure}

Figure 4 shows the wave heights calculated by ARTEMIS using
an incremental phase calculation.
The other requested graphical outputs (not shown here) are wave incidence, bathymetry and breaking ratio.

The goal is to obtain a domain that is not very sensitive to the choice of the phase because this data is
generally less well known than the wave height or the period.
Our recommendation is to choose liquid boundaries with fairly homogeneous bathymetries, avoiding imposing the incident
swell too close to the harbor as it is the case in
example. Although it is preferable to ensure good physical consistency of the input data.

This example gives a concrete case of the use of Artemis, with random swell
monodirectional swell. The choice of the reflection coefficients as well as the angles of attack of the
are interesting for the user.
Moreover, we find in this test case examples for the treatment of the
phase.

