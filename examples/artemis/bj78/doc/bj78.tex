\chapter{Energy loss due to wave breaking: BJ78}

\section{Purpose}
The objective of this test case is to compare Artemis with the results of a
physical model in a case of wave energy loss due to  breaking waves. The
experimentation took place at Delft University of Technology [1] on a
21m x 80cm model. The objective isto validate the modeling of the dissipation
due to breaking waves (Battjes and Janssen approach for random swell).

The results obtained with Artemis are in good agreement with the experimental
results.

\section{Description of the test case}

The modeled case and its bathymetry can be visualized on figure 1.
The water height is 0.616 m at the channel entrance. The surge is taken into
account
(Battjes and Janssen approach) but there is no dissipation by friction on the
bottom.

\subsection{Geometry}

\begin{figure}[h]
\begin{center}
  \includegraphicsmaybe{[width=0.7\textwidth]}{config.png}
\end{center}
\caption{Configuration.}
\label{fig:bj78_bathy}
\end{figure}
The modeled channel is a 21 m long and 0.8 m wide rectangle.
The bathymetry is set at z = -0.616m at the channel entrance, then several
ramps are applied:

$$
\begin{array}{ll}
  x < L_1 & z = - 0 . 616 \\
  L_1 \le x \le L_1 + L_2 &  z = 0.05( x - L_1 ) - 0.616\\
  L_1 + L_2 < x < L_1 + L_2 + L_3 & z = - 0.025( x-L_1 -L_2 ) - 0.116 \\
L 1 + L 2 + L 3 \le x & z = 0.05 (x-L_1-L_2-L_3) - 0.226 \\
\end{array}
$$



\subsection{Mesh}
\begin{figure}[h]
\begin{center}
  \includegraphicsmaybe{[width=0.7\textwidth]}{../img/Mesh.png}
\end{center}
\caption{Mesh.}
\label{fig:bj78_mesh}
\end{figure}

The mesh is a set mesh comprising 1394 triangular elements. The nodes are
spaced in x and y by about 16 cm.

\subsection{Boundary conditions}
\begin{figure}[h]
\begin{center}
  \includegraphicsmaybe{[width=0.7\textwidth]}{bj78_bc.png}
\end{center}
\caption{Boundary conditions.}
\label{fig:bj78_bc}
\end{figure}

Incidental swell:
\begin{itemize}
\item Peak period= 1.887 s
\item Significant wave height : 0.202 m
\item Min and Max period of the spectrum: [0.75 s ; 8 s]
\item JONSWAP spectrum parameter: gamma=3.3
\item Propagation direction : 0$^\circ$.
\item Direction of exit: 0$^\circ$.
\item Phase (ALFAP): zero on the whole border
\end{itemize}

Walls :
\begin{itemize}
\item Reflection coefficient = 1.
\item Phase shift : 0$^\circ$.
\item Angle of attack: 90$^\circ$.
\end{itemize}

Output :
\begin{itemize}
\item Angle of attack of the outgoing waves: 0$^\circ$.
\end{itemize}

\section{Reference solution}

The experimental results obtained in \cite{Battjes1978} constitute the reference values.
The criterion for comparison is the comparison is the energetic water height along the
channel. This water height is related to the significant head by the formula :
$$
H_s = \sqrt{2} H_e
$$
The measured and calculated free surface elevations are therefore compared.

\section{ Results}

The results are presented in figure \ref{fig:bj78_res}.
\begin{figure}[h]
\begin{center}
  \includegraphicsmaybe{[width=0.7\textwidth]}{resbj78.png}
\end{center}
\caption{Energetic wave height (m) as a function of the position in the channel (m).
Red curve: ARTEMIS, green points: Battjes and Janssen experiment \cite{Battjes1978}.}
\label{fig:bj78_res}
\end{figure}

The Artemis results correlate well with the experiment. The energy loss due to the
is well predicted in the case of a random swell. The code gives satisfaction in terms of
in terms of intensity and spatial distribution of this dissipation, the use of the
formula of

Battjes and Janssen's formula is a good way to model the dissipation related to the
breaking resulting from bathymetric variations.




