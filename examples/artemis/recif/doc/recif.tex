\chapter{Schematic reef: recif}

\section{Summary}

The objective of this test case is to validate, on a simple case,the use of the terms of
steep slope 
and curvature terms in the Berkhoff equation (diffraction-refraction equation extended to
to strong variations of bathymetry, option {\it RAPIDLY VARYING TOPOGRAPHY} option).
Artemis is compared to the results of \cite{Massel1993} and REEF2000 \cite{Michel1999}.

We consider the case of a channel with analytical bathymetry. This case was defined by
\cite{Massel1993} to represent the configuration of a submerged coral reef, without discontinuity
for curvature or slope. It allows to highlight significant deviations between the
predictions of the Berkhoff equation alone and the extended Berkhoff equation.
We also present the results obtained without the slope and curvature terms in order to
validate the treatment of the Berkhoff equation alone.

In both cases, the Artemis results are excellent.

\section{Description of the test case}

The general configuration is described in Figure \ref{fig:recif_config}, which shows the bathymetry and dimensions of the
channel.

\begin{figure}[h]
\begin{center}
  \includegraphicsmaybe{[width=0.7\textwidth]}{config.png}
\end{center}
\caption{Configuration of the problem}
\label{fig:recif_config}
\end{figure}

Several bathymetries (i.e. "b" values) are tested for validation. They are described in
in \ref{recif:geometry} By default, the test case file sets b = 2 m. The user is free to modify this value.
value.

The test case is chosen to be as simple as possible, so the calculation does not include either breaking or
friction of the bottom.


\subsection{Geometry}
\label{recif:geometry}
The modeled channel is a 40 m long and 2 m wide rectangle.
The bathymetry is set at z= 0 at the entrance of the channel, then a tangent
hyperbolic walk is applied via the following analytical bathymetry.
$$
z=\frac{d}{2}( 1+ \tanh (3 \pi (\frac{x-L1}{b}-0.5) ) )
$$
The variation of bathymetry is thus mainly on the domain " b ", outside it is
negligible.

We vary b from 20 cm to 20 m to appreciate the impact of the slope and curvature terms.
The coast of the free surface is 6m.


\subsection{Mesh size}

The mesh used is an adjusted mesh generated by Janet. It consists of 400 000
triangular elements. The nodes are regularly spaced by 2 cm on the x-axis and y-axis.

\subsection{Boundary conditions}

The boundary conditions of the domain are presented in Figure \ref{fig:recif_bc}.

\begin{figure}[h]
\begin{center}
  \includegraphicsmaybe{[width=0.7\textwidth]}{bc_recif.png}
\end{center}
\caption{Boudary conditions}
\label{fig:recif_bc}
\end{figure}

Incident swell:
\begin{itemize}
\item Period = 6.34 s
\item Wave height : 0.05 m
\item Propagation direction: 0$^\circ$.
\item Exit direction: 0$^\circ$.
\item Phase (ALPHAP): 0$^\circ$ on the whole liquid boundary
\end{itemize}
Wall:
\begin{itemize}
\item Reflection coefficient = 1.
\item Phase shift: 0$^\circ$
\item Angle of attack: 0$^\circ$
\end{itemize}
Free exit:
\begin{itemize}
\item Angle of attack of the outgoing waves: 0$^\circ$.
\end{itemize}

\section{Reference solution}

The results obtained by \cite{Massel1993} constitute the reference points for the following values of b
values: 0.4m, 0.5m, 1m, 2m, 4m, 8m.

In \cite{Michel1999}, we find curves describing all values of b in the interval [0.2m ;20m].

In these two references, the results obtained with the simple Berkhoff equation and with the
the extended Berkhoff equation are presented.
The comparison criterion used is the reflection coefficient, which we define as
as follows:

$$
C_{Reflexion}=\frac{H_{max}-H_{incident}}{H_{incident}}
$$

with $H_{incident}$ the incident wave height and
$H_{max}$ the maximum wave height observed in the resulting wave field (incident
+reflected)


\section{Results}

\begin{figure}[h]
\begin{center}
  \includegraphicsmaybe{[width=0.9\textwidth]}{../img/WaveHeight.png}
\end{center}
\caption{wave height for b=4m}
\label{fig:recif_resu_hm0}
\end{figure}

\begin{figure}[h]
\begin{center}
  \includegraphicsmaybe{[width=0.9\textwidth]}{resu_recif.png}
\end{center}
\caption{Reflection coefficients as a function of bathymetry (length b of the reef).
  Comparison between ARTEMIS (blue and grey points) with references \cite{Michel1999}
  (black points, blue diamonds) and \cite{Massel1993} (blue and black lines)}
\label{fig:recif_resu_reflec}
\end{figure}

The results obtained with Artemis are in very good agreement with the two references
bibliographic references (numerical). This is true for the Berkhoff equation alone or the extended equation
taking into account the effects of slope and curvature.

\section{Conclusions}

This test case allowed us to compare Artemis to other codes solving the Berkhoff equation
equation with or without taking into account slope and curvature effects. The results are excellent:
Artemis gives results very close to the two other codes on the whole range of cases
studied.
