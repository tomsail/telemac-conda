\chapter{Schematic port of Delft:port}

\section{Summary}

The objective of this test case is to compare Artemis with the results of a
physical model of a model of a schematic harbor configuration. The
experimentation took place at the Delft Hydraulics Laboratory Hydraulics
Laboratory in Delft [1] on a 10m x 8m model. The objective is to validate the
port agitation, in particular the taking into account of the reflection on
solid boundary.

The results obtained with Artemis are in good agreement with the experimental
results.

\section{Description of the test case}

\subsection{Geometry}

The modeled case is included in a rectangle of 10 m by 8 m. The entrance to
the harbor is 3 m. The water height is 30 cm and the bathymetry is zero over
the whole area.

\subsection{Mesh}

The mesh uses 3265 triangular elements, with a characteristic size of approximately 0.2 m.
It is presented in Figure \ref{fig:port_mesh}


\subsection{Mesh}
\begin{figure}[h]
\begin{center}
  \includegraphicsmaybe{[width=0.7\textwidth]}{port_mesh.png}
\end{center}
\caption{ Mesh used and measurement planes (sections) for comparisons
  Experimental/Digital}
\label{fig:port_mesh}
\end{figure}

\subsection{Boundary conditions}

The boundary conditions of the domain are presented in figure \ref{fig:port_bc},
all are solid wall type except for the harbor entrance, which carries an incident wave condition.

\begin{figure}[h]
\begin{center}
  \includegraphicsmaybe{[width=0.7\textwidth]}{port_bc.png}
\end{center}
\caption{ Defintition of the boundaries}
\label{fig:port_bc}
\end{figure}


Incidental swell:
\begin{itemize}
\item Period = 1.4 s
\item Wave height : 0.04 m
\item Direction of propagation : 0$^\circ$.
\item Exit direction : 0$^\circ$.
\item Phase (ALFAP) : 0$^\circ$ on the whole boundary
\end{itemize}
Wall 1:
\begin{itemize}
\item Reflection coefficient = 0.23
\item Phase shift : 0$^\circ$
\item Angle of attack: 0$^\circ$
\end{itemize}
Wall 2:- Reflection coefficient = 1.
\begin{itemize}
\item Phase shift : 0$^\circ$
\item Angle of attack: 0$^\circ$.
\end{itemize}
Wall 3:
\begin{itemize}
\item Reflection coefficient = 1.
\item Phase shift : 0$^\circ$
\item Angle of attack : 0$^\circ$.
\end{itemize}
Wall 4:
\begin{itemize}
\item Reflection coefficient = 0.05
\item Phase shift : 0$^\circ$
\item Angle of attack : 45$^\circ$
\end{itemize}
Wall 5:
\begin{itemize}
\item Reflection coefficient = 0.05
\item Phase shift : 0$^\circ$
\item Angle of attack : 0$^\circ$
\end{itemize}
Wall 6:
\begin{itemize}
\item Reflection coefficient = 0.23
\item Phase shift : 0$^\circ$
\item Angle of attack : 0$^\circ$
\end{itemize}

We note that a second case has been experimentally realized, with reflection coefficients
of 1 on all solid boundaries.

\section{Reference solution}

The results obtained in \cite{wavepenetration81} constitute the reference values, divided into 10 planes.
We compare the wave heights obtained numerically and experimentally by
presenting the amplification coefficient, defined by :
$$
\frac{H(x,y)}{H_incident}
$$

\section{Results}

\begin{figure}[h]
\begin{center}
  \includegraphicsmaybe{[width=0.9\textwidth]}{../img/WaveHeight.png}
\end{center}
\caption{ Wave height obtained with the last version of ARTEMIS}
\label{fig:port_waveheight}
\end{figure}


\begin{figure}[h]
\begin{center}
  \includegraphicsmaybe{[width=0.9\textwidth]}{port_resu.png}
\end{center}
\caption{ dimensionless water heights. Comparison between experimental measurements
(triangles) and ARTEMIS (lines)(Rq: results from the validation documents of version 3. update required).}
\label{fig:port_resu}
\end{figure}


The results obtained with Artemis are in good agreement with the experimental results. This
shows that the reflection coefficients are well taken into account by Artemis.


\section{Conclusions}

This test case compares Artemis with experimental results on physical model. It allows 
to validate the modeling of the harbor agitation, and in particular the treatment of
reflection coefficients on the solid boundaries in Artemis. The agreement between numerical predictions
and the experiment is satisfactory.


